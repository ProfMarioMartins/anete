% Options for packages loaded elsewhere
\PassOptionsToPackage{unicode}{hyperref}
\PassOptionsToPackage{hyphens}{url}
%
\documentclass[
  12pt,
]{article}
\usepackage{amsmath,amssymb}
\usepackage{iftex}
\ifPDFTeX
  \usepackage[T1]{fontenc}
  \usepackage[utf8]{inputenc}
  \usepackage{textcomp} % provide euro and other symbols
\else % if luatex or xetex
  \usepackage{unicode-math} % this also loads fontspec
  \defaultfontfeatures{Scale=MatchLowercase}
  \defaultfontfeatures[\rmfamily]{Ligatures=TeX,Scale=1}
\fi
\usepackage{lmodern}
\ifPDFTeX\else
  % xetex/luatex font selection
    \setmainfont[]{Roboto}
\fi
% Use upquote if available, for straight quotes in verbatim environments
\IfFileExists{upquote.sty}{\usepackage{upquote}}{}
\IfFileExists{microtype.sty}{% use microtype if available
  \usepackage[]{microtype}
  \UseMicrotypeSet[protrusion]{basicmath} % disable protrusion for tt fonts
}{}
\makeatletter
\@ifundefined{KOMAClassName}{% if non-KOMA class
  \IfFileExists{parskip.sty}{%
    \usepackage{parskip}
  }{% else
    \setlength{\parindent}{0pt}
    \setlength{\parskip}{6pt plus 2pt minus 1pt}}
}{% if KOMA class
  \KOMAoptions{parskip=half}}
\makeatother
\usepackage{xcolor}
\usepackage[margin=2.5cm]{geometry}
\usepackage{graphicx}
\makeatletter
\newsavebox\pandoc@box
\newcommand*\pandocbounded[1]{% scales image to fit in text height/width
  \sbox\pandoc@box{#1}%
  \Gscale@div\@tempa{\textheight}{\dimexpr\ht\pandoc@box+\dp\pandoc@box\relax}%
  \Gscale@div\@tempb{\linewidth}{\wd\pandoc@box}%
  \ifdim\@tempb\p@<\@tempa\p@\let\@tempa\@tempb\fi% select the smaller of both
  \ifdim\@tempa\p@<\p@\scalebox{\@tempa}{\usebox\pandoc@box}%
  \else\usebox{\pandoc@box}%
  \fi%
}
% Set default figure placement to htbp
\def\fps@figure{htbp}
\makeatother
\setlength{\emergencystretch}{3em} % prevent overfull lines
\providecommand{\tightlist}{%
  \setlength{\itemsep}{0pt}\setlength{\parskip}{0pt}}
\setcounter{secnumdepth}{-\maxdimen} % remove section numbering
\ifLuaTeX
\usepackage[bidi=basic]{babel}
\else
\usepackage[bidi=default]{babel}
\fi
\babelprovide[main,import]{brazilian}
\ifPDFTeX
\else
\babelfont{rm}[]{Roboto}
\fi
% get rid of language-specific shorthands (see #6817):
\let\LanguageShortHands\languageshorthands
\def\languageshorthands#1{}
\usepackage{bookmark}
\IfFileExists{xurl.sty}{\usepackage{xurl}}{} % add URL line breaks if available
\urlstyle{same}
\hypersetup{
  pdftitle={Análise e Expressão Textual (Anete)},
  pdfauthor={Mário Martins - UFERSA},
  pdflang={pt-BR},
  hidelinks,
  pdfcreator={LaTeX via pandoc}}

\title{Análise e Expressão Textual (Anete)}
\author{Mário Martins - UFERSA}
\date{}

\begin{document}
\maketitle

\section[Teste diagnóstico de habilidades
textuais]{\texorpdfstring{Teste diagnóstico de habilidades
textuais\footnote{Este teste diagnóstico de competências textuais está
  fundamentado nas diretrizes do Parecer CNE/CES nº 266/2011, que
  estabelece como competência do egresso de um bacharelado
  interdisciplinar ``ter sensibilidade social e aptidão para a
  comunicação''. Também se alinha à Resolução CNE nº 2/2019, referente
  aos cursos de engenharia, que prevê como competência do engenheiro
  ``expressar-se adequadamente na língua pátria (ou em outro idioma),
  inclusive por meio das tecnologias digitais de informação e
  comunicação''.}}{Teste diagnóstico de habilidades textuais}}\label{teste-diagnuxf3stico-de-habilidades-textuais1}

\vspace{1.0cm}

\textbf{Nome:}
\_\_\_\_\_\_\_\_\_\_\_\_\_\_\_\_\_\_\_\_\_\_\_\_\_\_\_\_\_\_\_\_\_\_\_\_\_\_\_\_\_\_\_\_\_\_\_\_\_\_\_\_\_\_\_\_\_\_\_\_\_\_\_\_\_\_\_\_\_\_\_\_\_\_\_\_\_\_\_\_\_\_\_\_\_\_\\
\vspace{0.5cm}

\textbf{Matrícula:}
\_\_\_\_\_\_\_\_\_\_\_\_\_\_\_\_\_\_\_\_\_\_\_\_\_\_\_\_\_\_\_\_\_\_\_\_\_\_\_\_\_\_\_\_\_\_\_\_\_\_\_\_\_\_\_\_\_\_\_\_\_\_\_\_\_\_\_\_\_\_\_\_\_\_\_\_\_\_\_\_\_\\
\vspace{0.5cm}

Este teste tem caráter \textbf{diagnóstico} e \textbf{não será utilizado
para nota}. Seu objetivo é conhecer melhor suas habilidades de leitura,
escrita e reescrita acadêmicas.

Leia com atenção cada proposta e responda com calma, da melhor forma que
puder. Ao final, revise suas respostas antes de entregar. Esta é apenas
uma atividade para conhecermos melhor o seu ponto de partida --- não se
preocupe com nota.

\begin{center}\rule{0.5\linewidth}{0.5pt}\end{center}

\subsection{Parte 1 -- Escrita}\label{parte-1-escrita}

\textbf{1. Escreva um parágrafo entre 100 e 150 palavras em que você
responda à seguinte pergunta:}

\textbf{``Por que a comunicação escrita é uma habilidade essencial no
ensino superior?''}

\begin{quote}
\footnotesize
Ao escrever o parágrafo, mencione pelo menos dois aspectos importantes da escrita e dê um exemplo prático de situação acadêmica em que ela se faz necessária. Escreva em linguagem formal, típica de contextos acadêmicos.
\normalsize
\end{quote}

\noindent\fbox{%
    \parbox[t][20cm][t]{\textwidth}{%
        \vspace{0.2cm}
        \textit{Espaço reservado para a resposta da parte 1 deste teste. Escreva dentro deste quadro.}
    }%
}

\subsection{}\label{section}

\vspace{0.25cm}

\subsection{Parte 2 -- Leitura}\label{parte-2-leitura}

Leia o texto a seguir com atenção e, em seguida, responda às questões
objetivas. \vspace{0.25cm}

\Large\textbf{Seja um engenheiro que escreve bem}

\small Daniel Ferraz\footnote{Artigo publicado no LinkedIn. Disponível
  em:
  \url{https://pt.linkedin.com/pulse/seja-um-engenheiro-que-escreve-bem-daniel-ferraz}}

\begin{quote}
Não gosto de café muito quente. Sempre quando vou tomar tenho que esperar alguns minutos até perder o medo de queimar minha boca. Enquanto meu café esfria, gosto de sentir o aroma que sobe lentamente, observando as gotículas de água condensadas na borda da xícara. É um fenômeno intrigante para mim.

Mas o que isso tem a ver com este artigo? Pouca coisa.

Vou contornar o assunto totalmente agora. Se no primeiro parágrafo desse artigo você já ficou se perguntando onde ele quer chegar e o que café está relacionado com um engenheiro comunicador, eu atingi meu objetivo. Queria mostrar que nós engenheiros, conseguimos nos comunicar, e falar sobre qualquer coisa, captar um leitor, transmitir uma mensagem.

Comunicação escrita é a forma como nos relacionamos aqui pelo Linkedin e em tantos outros lugares. Quando você abriu este artigo, entrou em um processo comunicativo comigo. O poder maior que podemos obter aqui no Linkedin não é o de disparar currículos com uma mensagem padronizada, o que de melhor pode acontecer com você é se tornar \textbf{um bom comunicador}.\\

\textbf{Por que escrevemos pouco?}

A comunicação escrita efetiva não foi ensinada na faculdade. Muito pelo contrário, foi desestimulada. Nunca vi uma prova de cálculo diferencial integral com uma questão dissertativa, onde você tivesse que explicar como uma integral tripla define o volume de um sólido qualquer.

Quem gostava de ler e/ou escrever na faculdade foi dogmatizado para ser fiel aos números, e não viver um adultério com as letras.

A impressão que tenho, é que alguns engenheiros são preguiçosos para escrever. Recebo dezenas de e-mails mensais com pedidos de empregos ou estágios, mas nunca recebi um sequer que fugisse do padrão superficial adotado por todos. Falta originalidade ou é uma displicência característica? Por que olhamos com enfado e desdém para a figura de uma máquina de escrever, a mãe dos teclados que você utiliza diariamente?

Não é que achamos uma máquina de escrever obsoleta, é que não gostamos do produto dela: o texto.

O fato é que poucos engenheiros escrevem bem. Uma timeline, independente da rede social que for, dá esse poder incrível ao interlocutor de ser notado. Um texto certo, com o conteúdo certo, para o leitor certo, e pronto! Eis o resultado que você esperava. Estabelecemos um relacionamento onde tudo pode acontecer.

É uma relação de causa e efeito. Consegue oportunidades quem é notado, é notado quem faz algo diferente, e um dos maiores diferenciais hoje é o engenheiro que escreve bem, sobretudo a respeito de engenharia.\\

\textbf{Não tenha medo de interação}

Questione, opine, comente, agradeça, compartilhe, discorde, ou seja, interaja!

Ensine e aprenda. Não seja um espírito invisível vagando pela rede, ou como aquela senhora do interior que passa a tarde toda observando o movimento na rua. Participe escrevendo.

Dê sua cara a tapa. Esse poder de ser ouvido (ou lido), por dezenas de milhares, e até centenas de milhares de pessoas, pode alavancar resultados não esperados.

Não será um exercício fácil, mas como tudo na vida a prática leva à perfeição.

Um bom texto a todos!

\end{quote}

\subsubsection{Questões de múltipla
escolha}\label{questuxf5es-de-muxfaltipla-escolha}

\textbf{2.1 -- O autor começa seu texto com uma descrição sobre o café.
Qual é a principal função dessa introdução?}\\
- ( ) a) Relatar um hábito cotidiano para criar empatia com o leitor\\
- ( ) b) Apresentar, de forma indireta, a importância da engenharia
térmica\\
- ( ) c) Causar estranhamento e prender a atenção, antes de introduzir o
tema principal\\
- ( ) d) Mostrar que o autor valoriza o cotidiano como objeto técnico de
reflexão

\vspace{0.5cm}

\textbf{2.2 -- O autor afirma que engenheiros escrevem pouco. Qual das
opções resume melhor sua explicação para esse fato?}\\
- ( ) a) Falta de tempo para desenvolver a escrita formal\\
- ( ) b) Formação acadêmica que valoriza apenas os números\\
- ( ) c) Pouca valorização do texto na cultura digital contemporânea\\
- ( ) d) Inexistência de disciplinas de escrita nas engenharias

\vspace{0.5cm}

\textbf{2.3 -- Em sua crítica, o autor usa a metáfora ``adultério com as
letras''. O que ela expressa?}\\
- ( ) a) Que a escrita é vista como algo proibido para quem é de
exatas\\
- ( ) b) Que a leitura e a escrita são práticas ultrapassadas na
engenharia\\
- ( ) c) Que os engenheiros evitam se envolver com o universo das letras
por preconceito formativo\\
- ( ) d) Que as humanidades não devem invadir o campo técnico da
engenharia

\vspace{0.5cm}

\textbf{2.4 -- Qual alternativa expressa melhor a tese do texto?}\\
- ( ) a) A comunicação escrita não tem mais espaço no mercado de
trabalho\\
- ( ) b) O LinkedIn é a principal rede para engenheiros que buscam
visibilidade\\
- ( ) c) Um engenheiro que escreve bem se destaca em sua área
profissional\\
- ( ) d) As redes sociais diminuíram o interesse dos engenheiros pela
escrita

\vspace{0.5cm}

\textbf{2.5 -- A imagem da ``senhora do interior que observa o movimento
na rua'' representa:}\\
- ( ) a) Uma atitude tradicional de vigilância social\\
- ( ) b) Um tipo de participação passiva nas redes digitais\\
- ( ) c) O comportamento reflexivo de quem observa antes de agir\\
- ( ) d) A nostalgia por práticas mais analógicas de comunicação

\vspace{0.5cm}

\textbf{2.6 -- Segundo o texto, qual é a relação entre escrever bem e se
destacar no mercado?}\\
- ( ) a) Quem escreve bem consegue notas mais altas na faculdade\\
- ( ) b) Escrever bem permite preencher currículos de forma mais clara\\
- ( ) c) A boa escrita é uma forma de se tornar visível e gerar
oportunidades profissionais\\
- ( ) d) Quem escreve bem tende a ocupar cargos gerenciais
automaticamente

\subsection{}\label{section-1}

\subsection{Parte 3 -- Reescrita}\label{parte-3-reescrita}

\textbf{3. O parágrafo abaixo apresenta problemas de clareza, coesão,
concordância e pontuação. Reescreva-o mais adequada, clara e
objetivamente, observando os padrões formais da escrita acadêmica.}

\textbf{Atenção:} a \textbf{reescrita deve preservar o sentido original}
do texto.

\begin{quote}
O acadêmico que entra na universidade precisa saber que escrever é uma
ferramenta que não pode ser deixada de lado por que muitos textos são
necessários serem produzidos durante o curso. E que escrever não é só
para português, mas para todas as matérias que precisa de relatório.
Porque se não escrever bem, como vai conseguir fazer um bom projeto de
pesquisa? Então, o ideal é melhorar isso logo no início do curso, pois
quem não escreve direito terá dificuldades para aprender outras coisas.
\end{quote}

\noindent\fbox{%
    \parbox[t][10cm][t]{\textwidth}{%
        \vspace{0.2cm}
        \textit{Espaço reservado para a resposta da parte 3 deste teste. Escreva dentro deste quadro.}
    }%
}

\vspace{1.0cm}

\textbf{3.1 Após reescrever, informe uma modificação importante que você
fez no texto e justifique sua escolha:}

\noindent\fbox{%
    \parbox[t][8cm][t]{\textwidth}{%
        \vspace{0.2cm}
        \textit{Exemplo: “Reformulei a primeira frase para evitar ambiguidade.”}
    }%
}

\end{document}

% Options for packages loaded elsewhere
% Options for packages loaded elsewhere
\PassOptionsToPackage{unicode}{hyperref}
\PassOptionsToPackage{hyphens}{url}
\PassOptionsToPackage{dvipsnames,svgnames,x11names}{xcolor}
%
\documentclass[
  letterpaper,
  DIV=11,
  numbers=noendperiod]{scrreprt}
\usepackage{xcolor}
\usepackage{amsmath,amssymb}
\setcounter{secnumdepth}{5}
\usepackage{iftex}
\ifPDFTeX
  \usepackage[T1]{fontenc}
  \usepackage[utf8]{inputenc}
  \usepackage{textcomp} % provide euro and other symbols
\else % if luatex or xetex
  \usepackage{unicode-math} % this also loads fontspec
  \defaultfontfeatures{Scale=MatchLowercase}
  \defaultfontfeatures[\rmfamily]{Ligatures=TeX,Scale=1}
\fi
\usepackage{lmodern}
\ifPDFTeX\else
  % xetex/luatex font selection
\fi
% Use upquote if available, for straight quotes in verbatim environments
\IfFileExists{upquote.sty}{\usepackage{upquote}}{}
\IfFileExists{microtype.sty}{% use microtype if available
  \usepackage[]{microtype}
  \UseMicrotypeSet[protrusion]{basicmath} % disable protrusion for tt fonts
}{}
\makeatletter
\@ifundefined{KOMAClassName}{% if non-KOMA class
  \IfFileExists{parskip.sty}{%
    \usepackage{parskip}
  }{% else
    \setlength{\parindent}{0pt}
    \setlength{\parskip}{6pt plus 2pt minus 1pt}}
}{% if KOMA class
  \KOMAoptions{parskip=half}}
\makeatother
% Make \paragraph and \subparagraph free-standing
\makeatletter
\ifx\paragraph\undefined\else
  \let\oldparagraph\paragraph
  \renewcommand{\paragraph}{
    \@ifstar
      \xxxParagraphStar
      \xxxParagraphNoStar
  }
  \newcommand{\xxxParagraphStar}[1]{\oldparagraph*{#1}\mbox{}}
  \newcommand{\xxxParagraphNoStar}[1]{\oldparagraph{#1}\mbox{}}
\fi
\ifx\subparagraph\undefined\else
  \let\oldsubparagraph\subparagraph
  \renewcommand{\subparagraph}{
    \@ifstar
      \xxxSubParagraphStar
      \xxxSubParagraphNoStar
  }
  \newcommand{\xxxSubParagraphStar}[1]{\oldsubparagraph*{#1}\mbox{}}
  \newcommand{\xxxSubParagraphNoStar}[1]{\oldsubparagraph{#1}\mbox{}}
\fi
\makeatother


\usepackage{longtable,booktabs,array}
\usepackage{calc} % for calculating minipage widths
% Correct order of tables after \paragraph or \subparagraph
\usepackage{etoolbox}
\makeatletter
\patchcmd\longtable{\par}{\if@noskipsec\mbox{}\fi\par}{}{}
\makeatother
% Allow footnotes in longtable head/foot
\IfFileExists{footnotehyper.sty}{\usepackage{footnotehyper}}{\usepackage{footnote}}
\makesavenoteenv{longtable}
\usepackage{graphicx}
\makeatletter
\newsavebox\pandoc@box
\newcommand*\pandocbounded[1]{% scales image to fit in text height/width
  \sbox\pandoc@box{#1}%
  \Gscale@div\@tempa{\textheight}{\dimexpr\ht\pandoc@box+\dp\pandoc@box\relax}%
  \Gscale@div\@tempb{\linewidth}{\wd\pandoc@box}%
  \ifdim\@tempb\p@<\@tempa\p@\let\@tempa\@tempb\fi% select the smaller of both
  \ifdim\@tempa\p@<\p@\scalebox{\@tempa}{\usebox\pandoc@box}%
  \else\usebox{\pandoc@box}%
  \fi%
}
% Set default figure placement to htbp
\def\fps@figure{htbp}
\makeatother





\setlength{\emergencystretch}{3em} % prevent overfull lines

\providecommand{\tightlist}{%
  \setlength{\itemsep}{0pt}\setlength{\parskip}{0pt}}



 


\KOMAoption{captions}{tableheading}
\makeatletter
\@ifpackageloaded{tcolorbox}{}{\usepackage[skins,breakable]{tcolorbox}}
\@ifpackageloaded{fontawesome5}{}{\usepackage{fontawesome5}}
\definecolor{quarto-callout-color}{HTML}{909090}
\definecolor{quarto-callout-note-color}{HTML}{0758E5}
\definecolor{quarto-callout-important-color}{HTML}{CC1914}
\definecolor{quarto-callout-warning-color}{HTML}{EB9113}
\definecolor{quarto-callout-tip-color}{HTML}{00A047}
\definecolor{quarto-callout-caution-color}{HTML}{FC5300}
\definecolor{quarto-callout-color-frame}{HTML}{acacac}
\definecolor{quarto-callout-note-color-frame}{HTML}{4582ec}
\definecolor{quarto-callout-important-color-frame}{HTML}{d9534f}
\definecolor{quarto-callout-warning-color-frame}{HTML}{f0ad4e}
\definecolor{quarto-callout-tip-color-frame}{HTML}{02b875}
\definecolor{quarto-callout-caution-color-frame}{HTML}{fd7e14}
\makeatother
\makeatletter
\@ifpackageloaded{caption}{}{\usepackage{caption}}
\AtBeginDocument{%
\ifdefined\contentsname
  \renewcommand*\contentsname{Table of contents}
\else
  \newcommand\contentsname{Table of contents}
\fi
\ifdefined\listfigurename
  \renewcommand*\listfigurename{List of Figures}
\else
  \newcommand\listfigurename{List of Figures}
\fi
\ifdefined\listtablename
  \renewcommand*\listtablename{List of Tables}
\else
  \newcommand\listtablename{List of Tables}
\fi
\ifdefined\figurename
  \renewcommand*\figurename{Figure}
\else
  \newcommand\figurename{Figure}
\fi
\ifdefined\tablename
  \renewcommand*\tablename{Table}
\else
  \newcommand\tablename{Table}
\fi
}
\@ifpackageloaded{float}{}{\usepackage{float}}
\floatstyle{ruled}
\@ifundefined{c@chapter}{\newfloat{codelisting}{h}{lop}}{\newfloat{codelisting}{h}{lop}[chapter]}
\floatname{codelisting}{Listing}
\newcommand*\listoflistings{\listof{codelisting}{List of Listings}}
\makeatother
\makeatletter
\makeatother
\makeatletter
\@ifpackageloaded{caption}{}{\usepackage{caption}}
\@ifpackageloaded{subcaption}{}{\usepackage{subcaption}}
\makeatother
\usepackage{bookmark}
\IfFileExists{xurl.sty}{\usepackage{xurl}}{} % add URL line breaks if available
\urlstyle{same}
\hypersetup{
  pdftitle={Análise e Expressão Textual},
  pdfauthor={Mário Martins},
  colorlinks=true,
  linkcolor={blue},
  filecolor={Maroon},
  citecolor={Blue},
  urlcolor={Blue},
  pdfcreator={LaTeX via pandoc}}


\title{Análise e Expressão Textual}
\author{Mário Martins}
\date{}
\begin{document}
\maketitle

\renewcommand*\contentsname{Table of contents}
{
\hypersetup{linkcolor=}
\setcounter{tocdepth}{2}
\tableofcontents
}

\chapter{Programação}\label{programauxe7uxe3o}

\chapter{Programação}\label{programauxe7uxe3o-1}

\section{Ementa}\label{ementa}

Abordar os paradigmas textuais e científicos na produção da escrita
científica, a intertextualidade como elemento de linguagem no contexto
da textualidade e da oralidade e da visualidade, a coesão e coerência
textual como elemento estruturador da linguagem acadêmica, o estilo como
mediador entre forma e conteúdo na produção do conhecimento, a
interdisciplinaridade como estética da linguagem.

\section{Objetivos}\label{objetivos}

Promover o desenvolvimento da competência comunicativa dos estudantes no
contexto acadêmico, por meio da leitura crítica, da análise linguística
e da produção de gêneros acadêmicos, com ênfase na escrita científica e
na elaboração de planos de trabalho.

Competência comunicativa é a capacidade de um indivíduo usar a língua de
forma eficaz e apropriada em diferentes contextos de comunicação. Esse
conceito vai além do simples domínio das regras gramaticais, englobando
também o conhecimento dos usos sociais da linguagem, a adequação ao
contexto, a capacidade de interpretar e produzir enunciados coerentes e
coesos, e de compreender as intenções dos interlocutores.

\section{Conteúdo Programático}\label{conteuxfado-programuxe1tico}

\subsection{Unidade I -- O mundo acadêmico e o fazer
científico}\label{unidade-i-o-mundo-acaduxeamico-e-o-fazer-cientuxedfico}

Fundamentos da vida universitária, da ciência e da escrita acadêmica • O
sistema acadêmico brasileiro: estrutura e funcionamento • Pensamento
científico • Fato, opinião e argumentação na escrita acadêmica • A
linguagem da ciência: características da escrita acadêmica • Gêneros
acadêmicos e o plano de trabalho como gênero institucional • Como fazer
citação e referência (normas ABNT) • Introdução à consciência
linguística: problemas da Olimpíada Brasileira de Linguística (OBL) •
Aspectos gramaticais: colocação pronominal e concordância verbal

\subsection{Unidade II -- Leitura, escrita e estilo na produção
científica}\label{unidade-ii-leitura-escrita-e-estilo-na-produuxe7uxe3o-cientuxedfica}

Desenvolvimento das competências de leitura, argumentação e escrita
científica • Leitura inspecional e averiguativa de textos científicos •
Fontes confiáveis e estratégias de busca de informação • Fichamento e
revisão de literatura • Coesão e coerência textual • Citações diretas e
indiretas; paráfrase e verbos dicendi • Estratégias de argumentação e
estilo na escrita acadêmica • Problemas da OBL e atividades de reflexão
linguística • Aspectos gramaticais: pontuação e acentuação

\subsection{Unidade III -- Produção do plano de
trabalho}\label{unidade-iii-produuxe7uxe3o-do-plano-de-trabalho}

Aplicação prática dos conhecimentos na redação de um gênero acadêmico
institucional • Etapas do projeto científico e estrutura do plano de
trabalho • A justificativa e os objetivos do projeto • Referencial
teórico, metodologia e cronograma • Redação clara, concisa e objetiva •
Revisão e reescrita com base em critérios técnicos • Avaliação final:
entrega do plano de trabalho • Atividades de revisão com base nos
objetivos e problemas da OBL • Diagnóstico linguístico final e
autoavaliação do progresso

\section{Competências e
habilidades}\label{competuxeancias-e-habilidades}

\begin{longtable}[]{@{}
  >{\raggedright\arraybackslash}p{(\linewidth - 2\tabcolsep) * \real{0.5000}}
  >{\raggedright\arraybackslash}p{(\linewidth - 2\tabcolsep) * \real{0.5000}}@{}}
\toprule\noalign{}
\begin{minipage}[b]{\linewidth}\raggedright
Competência
\end{minipage} & \begin{minipage}[b]{\linewidth}\raggedright
Habilidades
\end{minipage} \\
\midrule\noalign{}
\endhead
\bottomrule\noalign{}
\endlastfoot
Utilizar a linguagem com clareza, precisão e adequação nos contextos
acadêmicos. & - Reconhecer diferentes registros linguísticos e níveis de
formalidade.- Utilizar a norma-padrão da língua portuguesa em situações
de comunicação acadêmica. \\
Ler, compreender e interpretar textos acadêmicos de diferentes gêneros.
& - Identificar a organização composicional e os propósitos
comunicativos dos gêneros acadêmicos.- Desenvolver estratégias de
leitura crítica e analítica. \\
Produzir textos acadêmicos adequados aos gêneros e às normas
científicas. & - Elaborar textos com coesão e coerência.- Aplicar
corretamente normas de citação e referência (ABNT).- Planejar, redigir e
revisar textos científicos. \\
Mobilizar conhecimentos linguísticos, discursivos e culturais na leitura
e na produção de textos. & - Analisar os efeitos de sentido produzidos
por escolhas linguísticas e discursivas.- Reconhecer e empregar marcas
de intertextualidade e argumentação nos textos. \\
Desenvolver autonomia na produção científica e no uso das tecnologias da
informação. & - Utilizar ferramentas digitais para organização da
pesquisa e produção textual.- Elaborar um plano de trabalho segundo os
critérios de projetos de iniciação científica. \\
Ampliar a consciência linguística a partir da reflexão sobre o
funcionamento da língua. & - Resolver problemas linguísticos com base em
conhecimentos gramaticais.- Analisar fatos linguísticos em contextos
reais de uso. \\
\end{longtable}

\section{Metodologia}\label{metodologia}

A disciplina adota uma abordagem ativa e interdisciplinar, com
atividades práticas, estudos de caso, quizzes interativos, resolução de
problemas da Olimpíada Brasileira de Linguística (OBL), exercícios de
gramática, produção textual orientada, leitura e análise de textos e o
uso de ferramentas digitais, como editores de texto com revisão
automática e chats de IA para promover a revisão crítica da escrita.

\section{Avaliação}\label{avaliauxe7uxe3o}

Avaliação final (40\%): Produção de um plano de trabalho nos moldes
exigidos em editais de iniciação científica (Pibic).

Três provas objetivas (3 x 20\%): Realizadas ao final de cada unidade.

Plano de trabalho: seções esperadas

\begin{verbatim}
•   Identificação
•   Introdução e Justificativa
•   Objetivos
•   Geral
•   Específicos
•   Metodologia
•   Habilidades desenvolvidas
•   Referências bibliográficas
\end{verbatim}

\section{Bibliografia básica}\label{bibliografia-buxe1sica}

APPOLINÁRIO, Fabio; GIL, Isaac. Como escrever um texto científico. 1ª
ed.~{[}S. l.{]}: Trevisan Editora, 2013. Disponível em:
https://bookshelf.vitalsource.com/books/9788599519493.

CASTRO, Nádia Studzinski Estima de et al.~Leitura e escrita acadêmicas.
Porto Alegre: Sagah, 2019. Disponível em:
https://bookshelf.vitalsource.com/books/9788533500228.

VIEIRA, Francisco Eduardo; FARACO, Carlos Alberto. Escrever na
Universidade 1 - Fundamentos. 1ª ed.~{[}S. l.{]}: Parábola Editorial,
2019.

\section{Bibliografia complementar}\label{bibliografia-complementar}

ABNT NBR 10520:2023. Informação e documentação -- Citações em documentos
-- Apresentação. Rio de Janeiro: ABNT, 2023.

ABNT NBR 6023:2018. Informação e documentação -- Referências --
Elaboração. Rio de Janeiro: ABNT, 2018.

CUNHA, Celso; CINTRA, Lindley. Nova gramática do português
contemporâneo. 7. ed.~Rio de Janeiro: Lexikon, 2019. E-book. Disponível
em: https://plataforma.bvirtual.com.br. Acesso em: 02 abr. 2025.

FERRAREZZI JÚNIOR, Celso. Guia do trabalho científico: do projeto à
redação final. 1ª ed.~{[}S. l.{]}: Contexto, 2011. Disponível em:
https://bookshelf.vitalsource.com/books/9788572447638.

MENDES, Gilmar Ferreira; FOSTER JÚNIOR, Nestor José. Manual de redação
da Presidência da República. rev. atual. Brasília: Presidência da
República, 2002. Disponível em:
https://www4.planalto.gov.br/centrodeestudos/assuntos/manual-de-redacao-da-presidencia-da-republica/manual-de-redacao.pdf\#page=12.08.

Outras obras poderão ser usadas na disciplina.

\part{Apresentação}

\part{Unidade I}

\chapter{Sistema acadêmico
brasileiro}\label{sistema-acaduxeamico-brasileiro}

Roteiro de aula elaborado no RStudio com o auxílio da inteligência
artificial ChatGPT, revisado e avaliado pelo professor antes de sua
publicação.

\section{Objetivo de aprendizagem}\label{objetivo-de-aprendizagem}

Ao final deste encontro, e com base na leitura indicada, espera-se que
você seja capaz de:

\begin{itemize}
\tightlist
\item
  Identificar a proposta e a organização da disciplina Análise e
  Expressão Textual (Anete), reconhecendo seus objetivos, estrutura em
  unidades e formas de avaliação;
\item
  Compreender a organização do sistema acadêmico brasileiro,
  identificando suas principais instâncias, normas e formas de produção
  do conhecimento, especialmente os gêneros textuais acadêmicos;
\item
  Participar de um diagnóstico inicial de competência textual, visando à
  identificação de conhecimentos prévios e dificuldades na produção
  escrita em contexto universitário.
\end{itemize}

\emph{Leitura indicada:}\\
\textbf{Entendendo a organização do sistema acadêmico brasileiro},
capítulo do livro \emph{Guia do trabalho científico: do projeto à
redação final}, de Celso Ferrarezi Jr.

\href{vbk://9788572447638/epubcfi/6/14\%5B\%3Bvnd.vst.idref\%3Dcap1.xhtml\%5D}{Acesso
à leitura indicada}

\begin{center}\rule{0.5\linewidth}{0.5pt}\end{center}

\section{Aprendizagem prática}\label{aprendizagem-pruxe1tica}

\subsection{Questão 1}\label{questuxe3o-1}

Como se organiza o sistema acadêmico brasileiro?

\chapter{Pensamento científico}\label{pensamento-cientuxedfico}

Roteiro de aula elaborado no RStudio com o auxílio da inteligência
artificial ChatGPT, revisado e avaliado pelo professor antes de sua
publicação.

\section{Objetivos de aprendizagem}\label{objetivos-de-aprendizagem}

Ao final deste encontro e com base na leitura indicada, espera-se que
você seja capaz de:

\begin{itemize}
\tightlist
\item
  Reconhecer as características do pensamento científico,
  diferenciando-o de outras formas de pensar e compreender sua
  relevância na construção do saber universitário.
\end{itemize}

Leitura indicada: Think like a scientist: The power of a scientific
mindset, postagem do blog Ivory Embassy, por Santiago Gisler.

\href{https://ivoryembassy.com/scientific-mindset/}{Acesso à leitura
indicada}

\begin{center}\rule{0.5\linewidth}{0.5pt}\end{center}

\section{Aprendizagem prática}\label{aprendizagem-pruxe1tica-1}

\subsection{Questão 1}\label{questuxe3o-1-1}

Você pensa cientificamente?

\subsection{Questão 2}\label{questuxe3o-2}

O que é a pensamento científico?

\begin{quote}
O pensamento científico é um estilo de pensamento que incentiva a
análise crítica e o ceticismo na abordagem de situações e problemas.
\end{quote}

\begin{quote}
O pensamento científico promove o questionamento de suposições e a
avaliação objetiva de ideias, visando uma compreensão mais próxima da
verdade, na medida do possível.
\end{quote}

\begin{quote}
Por meio do uso de evidências e do raciocínio lógico, é possível
interpretar melhor o mundo ao redor, tomar decisões fundamentadas e
encontrar soluções para desafios.
\end{quote}

\href{https://www.dw.com/pt-br/teste-de-dna-resolve-enigma-de-virgem-maria-que-chorava-sangue/a-71703557}{Teste
de DNA resolve enigma de estátua que chorava sangue}

\subsection{Questão 3}\label{questuxe3o-3}

O que é fundamental no pensamento científico?

\begin{quote}
Um elemento fundamental no pensamento científico é o uso do método
científico.
\end{quote}

\begin{quote}
O método científico auxilia na aquisição de conhecimento por meio da
observação e da experiência (de forma empírica).
\end{quote}

\begin{quote}
O método científico organiza o processo pelo qual cientistas -- e
não-cientistas -- chegam a conclusões.
\end{quote}

\subsection{Questão 4}\label{questuxe3o-4}

Passo a passo do método científico (em síntese):

• Observar um fenômeno e formular uma pergunta • Realizar pesquisas e
reunir informações • Formular uma hipótese • Testar essa hipótese •
Apresentar os resultados, debater e aperfeiçoar o entendimento do
fenômeno

\subsection{Questão 5}\label{questuxe3o-5}

Pilares do pensamento científico (para além da ciência)

\begin{itemize}
\item[$\square$]
  \textbf{Demonstrar curiosidade:}\\
  \emph{Exemplo:} Um estudante de biologia que, ao observar uma planta
  com folhas amarelas, começa a se perguntar: ``Por que essa planta tem
  uma coloração diferente das outras? Será falta de nutrientes, excesso
  de sol, ou alguma doença?''
\item[$\square$]
  \textbf{Questionar pressupostos:}\\
  \emph{Exemplo:} Um jornalista investigativo que, ao receber um dado
  oficial, questiona: ``Será que esses números foram coletados
  corretamente? Quem financiou a pesquisa? Existe algum conflito de
  interesse?''
\item[$\square$]
  \textbf{Buscar evidências:}\\
  \emph{Exemplo:} Um consumidor que lê avaliações de múltiplos sites e
  compara estudos antes de comprar um produto que alega ser ecológico.
\item[$\square$]
  \textbf{Aplicar lógica e raciocínio:}\\
  \emph{Exemplo:} Um médico que, ao interpretar exames laboratoriais,
  descarta hipóteses menos prováveis com base nos padrões observados nos
  resultados.
\item[$\square$]
  \textbf{Adotar ceticismo saudável:}\\
  \emph{Exemplo:} Um economista que, ao ouvir que um novo investimento
  promete lucros altíssimos em pouco tempo, questiona: ``Quais são os
  riscos? Existe alguma evidência de que outros investidores tiveram
  sucesso semelhante?''
\item[$\square$]
  \textbf{Considerar explicações alternativas:}\\
  \emph{Exemplo:} Um pesquisador que, ao encontrar uma correlação entre
  duas variáveis, pensa: ``Será que é uma relação causal direta? Ou pode
  haver um terceiro fator influenciando ambas?''
\item[$\square$]
  \textbf{Estar aberto a novas ideias:}\\
  \emph{Exemplo:} Um professor que, ao descobrir um método de ensino
  inovador, considera aplicá-lo em suas aulas, mesmo que isso signifique
  mudar abordagens tradicionais.
\item[$\square$]
  \textbf{Refletir sobre suas próprias crenças:}\\
  \emph{Exemplo:} Um defensor de uma dieta específica que, ao encontrar
  novos estudos que contradizem suas crenças, analisa esses dados e
  ajusta suas recomendações.
\item[$\square$]
  \textbf{Buscar feedback e aprender com os erros:}\\
  \emph{Exemplo:} Um engenheiro que compartilha seu projeto com colegas,
  recebe críticas sobre a estrutura, faz alterações e apresenta um
  produto final mais eficiente.
\item[$\square$]
  \textbf{Valorizar a replicabilidade e a objetividade:}\\
  \emph{Exemplo:} Um cientista que publica os dados brutos de seu
  experimento e descreve os procedimentos em detalhes, permitindo que
  outros pesquisadores tentem reproduzir os mesmos resultados.
\end{itemize}

\begin{center}\rule{0.5\linewidth}{0.5pt}\end{center}

O funcionário da GM voltou nos dias seguintes e só variou o sabor do
sorvete. Mais uma vez, o carro só não pegava quando o sabor escolhido
era baunilha.

O problema acabou virando uma obsessão para o engenheiro, que fez
experiências diárias, anotou todos os detalhes possíveis e, depois de
duas semanas, chegou à primeira grande descoberta: quando escolhia
baunilha, o comprador gastava menos tempo, porque não precisava ficar
escolhendo o tipo de sorvete.

Examinando o carro, o engenheiro fez nova descoberta: com o tempo de
compra reduzido no caso da baunilha, em comparação com o tempo dos
outros sabores, o motor não chegava a esfriar. Com isso, os vapores de
combustível não se dissipavam, impedindo que a nova partida fosse
instantânea.

A partir deste episódio, a Pontiac mudou o sistema de alimentação de
combustível e introduziu a alteração em todos os modelos a partir desta
linha.

Mais que isso, o autor da reclamação ganhou um carro novo, além da
reforma do que não pegava com sorvete de baunilha.

A GM distribuiu também um memorando interno, exigindo que seus
funcionários levem a sério até as `reclamações mais estapafúrdias,
porque pode ser que uma grande inovação esteja por trás de um sorvete de
baunilha' diz a carta da GM.

\chapter{Opinião, fato e escrita
acadêmica}\label{opiniuxe3o-fato-e-escrita-acaduxeamica}

Roteiro de aula elaborado no RStudio com o auxílio da inteligência
artificial ChatGPT, revisado e avaliado pelo professor antes de sua
publicação.

\section{Objetivos de aprendizagem}\label{objetivos-de-aprendizagem-1}

Ao final deste encontro e com base na leitura indicada, espera-se que
você seja capaz de:

\begin{itemize}
\item
  Reconhecer e diferenciar exemplos de opiniões e fatos evidenciados em
  textos acadêmicos.
\item
  Reformular enunciados opinativos em afirmações sustentadas por
  evidências.
\item
  Analisar os recursos linguísticos que caracterizam afirmações
  opinativas, distinguindo-os das marcas típicas de afirmações factuais.
\end{itemize}

Leitura indicada: Uma questão de opinião, capítulo do livro Como
escrever um texto científico, de Fábio Appolinário e Isaac Gil.

\href{vbk://9788599519493/page/11}{Acesso à leitura indicada}

\begin{center}\rule{0.5\linewidth}{0.5pt}\end{center}

\section{Introdução}\label{introduuxe7uxe3o}

Como vimos anteriormente, o \textbf{pensamento científico} baseia-se em
uma \textbf{curiosidade ativa}, no \textbf{questionamento de
pressupostos} e na busca rigorosa por \textbf{evidências}, utilizando
\textbf{lógica} e \textbf{ceticismo} para avaliar as informações de
maneira \textbf{objetiva}. Além disso, considera \textbf{alternativas},
mantém-se \textbf{aberto a novas ideias} e promove a \textbf{reflexão
sobre as próprias crenças}, sempre valorizando a
\textbf{replicabilidade} e o aprendizado com \textbf{erros e feedbacks}.
Esses pilares, mais do que ferramentas científicas, são práticas que
enriquecem a \textbf{análise crítica} e a \textbf{tomada de decisões} em
diversas situações.

Esses pilares do pensamento científico não apenas fundamentam a
investigação científica, mas também guiam as \textbf{práticas de leitura
e escrita acadêmicas}. Essas competências intelectuais formam a base
para compreender, analisar e produzir textos acadêmicos de alta
qualidade. A leitura criteriosa permite identificar as pistas essenciais
nos textos, enquanto a escrita disciplinada e fundamentada em padrões
normativos promove a clareza, a coesão e a credibilidade das ideias.

O desenvolvimento do pensamento científico está intrinsecamente ligado
ao \textbf{domínio da linguagem acadêmica}, contribuindo para um
percurso mais autônomo e qualificado no ambiente universitário.

\subsection{Você sabe escrever e ler cientificamente? (aplicação de
teste
diagnóstico)}\label{vocuxea-sabe-escrever-e-ler-cientificamente-aplicauxe7uxe3o-de-teste-diagnuxf3stico}

\subsection{O que é uma escrita científica? Um texto científico? Texto
opinativo ou
factual?}\label{o-que-uxe9-uma-escrita-cientuxedfica-um-texto-cientuxedfico-texto-opinativo-ou-factual}

\begin{quote}
Segundo Appolinário e Gil, ``(\ldots) uma característica importante da
maioria das modalidades de textos científicos é exatamente a ausência da
opinião.'' p.~16
\end{quote}

\begin{quote}
``(\ldots) não há espaço para a enunciação de juízos de valor, que
possuem caráter opinativo ou especulativo.'' p.~16
\end{quote}

\begin{quote}
``(\ldots) o texto monográfico {[}leia-se científico{]} deve se ater a:

Descrição, comparação ou análise de teorias e conceitos desenvolvidos
por outros autores

Apresentação de dados secundários de pesquisa (dados coletados por
outros autores e devidamente citados no texto)

Apresentação e análise de dados primários (dados coletados pelo próprio
aluno e apresentados no trabalho)

p. 16
\end{quote}

\begin{quote}
Em síntese, o texto acadêmico é ``(\ldots) um exercício de argumentação
(desenvolvido em sua maior parte com as ``palavras'' do autor do
argumento), fortalecido tanto quanto possível por evidências retiradas
de teorias, conceitos e dados próprios ou de outrem, devidamente citados
no corpo do texto, de acordo com as normas adotadas pela instituição na
qual o aluno está desenvolvendo seu trabalho.'' p.~18
\end{quote}

\texttt{r\ emo::ji("warning")} \textbf{Será?}
\texttt{r\ emo::ji("warning")}

``(\ldots) não há informação pura, nem opinião girando no vazio.''
(Vieira; Faraco, 2019, p.~166)\footnote{VIEIRA, Francisco Eduardo;
  FARACO, Carlos Alberto. \textbf{Escrever na universidade} 2 - texto e
  discurso. São Paulo: Editora Parábola. 2019.}

\begin{center}\rule{0.5\linewidth}{0.5pt}\end{center}

\section{Aprendizagem prática}\label{aprendizagem-pruxe1tica-2}

\subsection{Questão 1}\label{questuxe3o-1-2}

Leia as frases abaixo e assinale FATO ou OPINIÃO.

\begin{center}\rule{0.5\linewidth}{0.5pt}\end{center}

\subsection{Questão 2}\label{questuxe3o-2-1}

Leia o parágrafo abaixo com atenção, extraído da obra ``Como escrever um
texto científico'', de Fábio Appolinário e Isaac Gil. Por que ele não
pode ser cientificamente adequado?

\begin{quote}
As empresas brasileiras têm enfrentado enormes problemas em função da
grande crise econômica pela qual passa nosso país, tendo assumido uma
atitude preventiva e cautelosa no que se refere aos investimentos,
notadamente no setor de treinamento -- que costuma ser o primeiro a
sentir os cortes em crises desse porte.
\end{quote}

Reescreva-o a fim de torná-lo cientificamente adequado.

\textbf{Sugestão de reescrita dos autores}

Segundo dados do Banco Central do Brasil (2009), em seis dos oito
trimestres dos anos de 2008 e 2009 houve uma retração do PIB brasileiro
entre 1,5\% e 2,3\%, o que, para alguns autores (por exemplo: KRAUSE,
2001; SILVEIRA, 2005), pode configurar uma ``crise econômica''. Nesse
mesmo período, em pesquisa exploratória realizada junto a gestores de
recursos humanos nas empresas do setor industrial da região sudeste do
país, a Associação Brasileira de Recursos Humanos (2009) relatou uma
intenção de cortes com gastos em treinamento de pessoal da ordem de
27,8\%, para o ano de 2010. É possível hipotetizar, portanto, que a
retração econômica tenha levado a uma diminuição na previsão de
investimentos nessa área.

\begin{center}\rule{0.5\linewidth}{0.5pt}\end{center}

\subsection{Questão 3}\label{questuxe3o-3-1}

Leia o texto abaixo e resolva as questões seguintes.\footnote{Fonte:
  Instituto Claro
  https://www.institutoclaro.org.br/educacao/para-aprender/roteiros-de-estudo/estudar-em-casa-diferenca-entre-fato-e-opiniao/}

\begin{center}\rule{0.5\linewidth}{0.5pt}\end{center}

\subsection{Questão 4}\label{questuxe3o-4-1}

Leia as frases abaixo e assinale FATO ou OPINIÃO\footnote{Fonte:
  Instituto Claro
  https://www.institutoclaro.org.br/educacao/para-aprender/roteiros-de-estudo/estudar-em-casa-diferenca-entre-fato-e-opiniao/}

\begin{center}\rule{0.5\linewidth}{0.5pt}\end{center}

\subsection{Questão 5}\label{questuxe3o-5-1}

Agora que você já sabe diferenciar uma afirmação opinativa de uma
afirmação factual, \textbf{em trios}, analisem novamente as frases
utilizadas nas atividades anteriores.

Para cada afirmação opinativa, identifiquem:

\begin{verbatim}
•   Adjetivos subjetivos (ex.: melhor, agradável, marcante);
•   Expressões com carga emocional ou afetiva (ex.: infelizmente, lamentável, encantava);
•   Ausência de dados verificáveis (indiquem onde faltam números, fontes ou evidências concretas).
\end{verbatim}

Para cada afirmação factual, indiquem:

\begin{verbatim}
•   Fontes de dados verificáveis (se presentes);
•   Tom neutro e objetivo (verifiquem se não há adjetivos subjetivos ou emoções envolvidas).
\end{verbatim}

\textbf{Marcas linguísticas da opinião}

As marcas linguísticas que caracterizam as afirmações opinativas
incluem: 1. \textbf{Uso de adjetivos avaliativos ou subjetivos}: Termos
como \emph{melhor}, \emph{mais agradável}, \emph{mais confortável},
\emph{dura}, entre outros, expressam julgamentos de valor ou
preferências. 2. \textbf{Presença de modalizadores e verbos de opinião}:
Expressões como \emph{eu acho}, \emph{é considerado}, \emph{é visto
como}, \emph{há quem sustente que} indicam a inserção de uma perspectiva
pessoal ou coletiva, sem compromisso com a objetividade. 3.
\textbf{Ausência de fontes confiáveis ou dados concretos}: Afirmações
opinativas frequentemente não apresentam referência a estudos,
estatísticas ou documentações que sustentem a afirmação. 4.
\textbf{Emprego de generalizações, suposições ou evidenciais
subjetivos}: Formulações como \emph{devem ser}, \emph{parecem}, \emph{é
frequentemente considerado}, \emph{segundo muitos} sugerem que a ideia
se apoia em impressões, crenças difusas ou fontes não identificadas. 5.
\textbf{Foco em emoções ou experiências pessoais}: Declarações que se
baseiam em preferências emocionais, como \emph{assistir filmes em casa é
muito mais agradável} ou \emph{chocolate é o melhor doce do mundo},
denotam um ponto de vista subjetivo.

Em contraposição, as afirmações factuais são marcadas por: • Dados
verificáveis e concretos; • Referência a fontes confiáveis; • Tom neutro
e objetivo, sem julgamentos de valor ou envolvimento emocional.

\subsection{Questão 6}\label{questuxe3o-6}

Ainda em trios, busquem no Google Acadêmico um artigo científico
publicado entre 2022 e 2025, com a palavra-chave ``ensino de
matemática''.

\begin{verbatim}
•   A pesquisa deve ser feita com os seguintes filtros:
•   Idioma: Português;
•   Tipo de documento: artigos (não incluir citações ou patentes);
\end{verbatim}

Escolham um artigo e analisem o resumo com base nos seguintes critérios:

\begin{verbatim}
•   Há fontes de dados verificáveis? (Ex.: menção a experimentos, estatísticas, estudos de caso, autores, instituições, etc.).
•   O tom do texto é neutro e objetivo? Ou há presença de adjetivos subjetivos ou carga emocional?
\end{verbatim}

Postem o resumo e a análise no Padlet da turma. No post, adicionem:

\begin{verbatim}
•   Título do artigo;
•   Resumo copiado;
•   Indicação das fontes verificáveis (se houver);
•   Comentário sobre o tom do texto (neutro ou subjetivo);
•   Trechos específicos que justifiquem a análise.
\end{verbatim}

🔗 Acesse o Padlet da turma
\href{https://padlet.com/mariomartins/anete-t01-b3r3eohs4gd7np8m}{aqui}

\texttt{r\ emo::ji("fire")} Você deve fazer login no Padlet para que a
postagem seja identificada.

Você está presente?

\chapter{Argumentos de senso comum e senso
crítico}\label{argumentos-de-senso-comum-e-senso-cruxedtico}

Roteiro de aula elaborado no RStudio com o auxílio da inteligência
artificial ChatGPT, revisado e avaliado pelo autor antes de sua
publicação.

\section{Objetivos de aprendizagem}\label{objetivos-de-aprendizagem-2}

Ao final deste encontro e com base na leitura indicada, espera-se que
você seja capaz de:

\begin{itemize}
\item
  Reconhecer a diferença entre argumentos baseados no senso comum e
  argumentos fundamentados no senso crítico.
\item
  Exercitar habilidades cognitivas necessárias à argumentação e ao
  pensamento científico.
\end{itemize}

Leitura indicada: Leitura e escrita acadêmicas, capítulo do livro
Leitura e escrita acadêmicas, de Nádia Castro e outros.

\href{vbk://9788533500228/page/11}{Acesso à leitura indicada}

\begin{center}\rule{0.5\linewidth}{0.5pt}\end{center}

\section{Introdução}\label{introduuxe7uxe3o-1}

Segundo Antônio Suárez Abreu, em seu livro ``A arte de argumentar:
gerenciando razão e emoção'' (2012, p.~42\footnote{ABREU, Antonio
  Suárez. A arte de argumentar: gerenciando razão e emoção. Cotia:
  Ateliê Editorial, 2001.}):

\begin{quote}
Argumentar (\ldots) não é tentar provar o tempo todo que temos razão,
impondo nossa vontade.
\end{quote}

\begin{quote}
Argumentar é, em primeiro lugar, convencer, ou seja, vencer junto com o
outro, caminhando ao seu lado, utilizando, com ética, as técnicas
argumentativas, para remover os obstáculos que impedem o consenso.''
\end{quote}

Ainda segundo o autor (2012, p.~42),

\begin{quote}
Argumentar é também saber persuadir, preocupar-se em ver o outro por
inteiro, ouvi-lo, entender suas necessidades, sensibilizar-se com seus
sonhos e emoções.
\end{quote}

Argumentar significa tanto convencer, como persuadir.

🧠 Convencer: levar o outro à aceitação de uma ideia por meio da razão,
da lógica e de evidências verificáveis.

❤️ Persuadir: levar o outro à aceitação de uma ideia por meio da emoção,
da empatia e da sensibilidade, respeitando sua autonomia.

Nádia Castro e outros autores (2019, p.~14) nos lembram que

\begin{quote}
Quando se pensa em argumentação, é preciso necessariamente remeter ao
caráter dialógico --- isto é, de diálogo --- dos discursos.
\end{quote}

\begin{quote}
Tudo aquilo que pensamos e fazemos é fruto dos discursos que nos
constroem enquanto seres psicossociais.
\end{quote}

Nesse caso, \textbf{discurso} significa um conjunto estruturado de
ideias, valores, crenças, formas de linguagem e modos de ver o mundo que
circulam socialmente e moldam nosso modo de pensar, agir e interpretar a
realidade.

\section{Aprendizagem prática}\label{aprendizagem-pruxe1tica-3}

\subsection{Questão 1}\label{questuxe3o-1-3}

Você consegue identificar que discursos são descritos na tabela abaixo?

Preencha as lacunas da primeira coluna com os nomes dos discursos.
Preencha também as lacunas da última coluna com dois ou três exemplos de
textos típicos de cada um dos discursos identificados por você.

\begin{longtable}[]{@{}
  >{\raggedright\arraybackslash}p{(\linewidth - 6\tabcolsep) * \real{0.0404}}
  >{\raggedright\arraybackslash}p{(\linewidth - 6\tabcolsep) * \real{0.6233}}
  >{\raggedright\arraybackslash}p{(\linewidth - 6\tabcolsep) * \real{0.2601}}
  >{\raggedright\arraybackslash}p{(\linewidth - 6\tabcolsep) * \real{0.0762}}@{}}
\caption{Discursos}\tabularnewline
\toprule\noalign{}
\begin{minipage}[b]{\linewidth}\raggedright
Discurso
\end{minipage} & \begin{minipage}[b]{\linewidth}\raggedright
Características centrais
\end{minipage} & \begin{minipage}[b]{\linewidth}\raggedright
Objetivo principal
\end{minipage} & \begin{minipage}[b]{\linewidth}\raggedright
Exemplos típicos
\end{minipage} \\
\midrule\noalign{}
\endfirsthead
\toprule\noalign{}
\begin{minipage}[b]{\linewidth}\raggedright
Discurso
\end{minipage} & \begin{minipage}[b]{\linewidth}\raggedright
Características centrais
\end{minipage} & \begin{minipage}[b]{\linewidth}\raggedright
Objetivo principal
\end{minipage} & \begin{minipage}[b]{\linewidth}\raggedright
Exemplos típicos
\end{minipage} \\
\midrule\noalign{}
\endhead
\bottomrule\noalign{}
\endlastfoot
? & Usa linguagem formal e objetiva; é baseado em evidências empíricas e
métodos sistemáticos; busca por generalização, explicação ou predição &
Produzir conhecimento racional, verificável e universal & ? \\
? & Usa linguagem técnica, normativa e prescritiva; é fortemente
influenciado por códigos e leis; depende de estrutura argumentativa
rígida & Regular comportamentos sociais com base na legislação & ? \\
? & Usa linguagem persuasiva, apelativa e ideológica; enfatiza valores,
identidades e crenças coletivas; visa gerar adesão ou mobilização &
Influenciar decisões, conquistar apoio e justificar ações & ? \\
? & Usa inguagem simbólica e valorativa; é baseado em doutrinas e
crenças; apela à fé, à moral e à transcendência & Promover uma visão de
mundo espiritual e normativa & ? \\
\end{longtable}

\begin{longtable}[]{@{}
  >{\raggedright\arraybackslash}p{(\linewidth - 6\tabcolsep) * \real{0.0418}}
  >{\raggedright\arraybackslash}p{(\linewidth - 6\tabcolsep) * \real{0.5285}}
  >{\raggedright\arraybackslash}p{(\linewidth - 6\tabcolsep) * \real{0.2205}}
  >{\raggedright\arraybackslash}p{(\linewidth - 6\tabcolsep) * \real{0.2091}}@{}}
\caption{Discursos}\tabularnewline
\toprule\noalign{}
\begin{minipage}[b]{\linewidth}\raggedright
Discurso
\end{minipage} & \begin{minipage}[b]{\linewidth}\raggedright
Características centrais
\end{minipage} & \begin{minipage}[b]{\linewidth}\raggedright
Objetivo principal
\end{minipage} & \begin{minipage}[b]{\linewidth}\raggedright
Exemplos típicos
\end{minipage} \\
\midrule\noalign{}
\endfirsthead
\toprule\noalign{}
\begin{minipage}[b]{\linewidth}\raggedright
Discurso
\end{minipage} & \begin{minipage}[b]{\linewidth}\raggedright
Características centrais
\end{minipage} & \begin{minipage}[b]{\linewidth}\raggedright
Objetivo principal
\end{minipage} & \begin{minipage}[b]{\linewidth}\raggedright
Exemplos típicos
\end{minipage} \\
\midrule\noalign{}
\endhead
\bottomrule\noalign{}
\endlastfoot
Científico & Usa linguagem formal e objetiva; é baseado em evidências
empíricas e métodos sistemáticos; busca por generalização, explicação ou
predição & Produzir conhecimento racional, verificável e universal &
Artigos científicos, relatórios de pesquisa \\
Jurídico & Usa linguagem técnica, normativa e prescritiva; é fortemente
influenciado por códigos e leis; depende de estrutura argumentativa
rígida & Regular comportamentos sociais com base na legislação &
Sentenças judiciais, leis, pareceres jurídicos \\
Político & Usa linguagem persuasiva, apelativa e ideológica; enfatiza
valores, identidades e crenças coletivas; visa gerar adesão ou
mobilização & Influenciar decisões, conquistar apoio e justificar ações
& Discursos eleitorais, campanhas, debates parlamentares \\
Religioso & Usa inguagem simbólica e valorativa; é baseado em doutrinas
e crenças; apela à fé, à moral e à transcendência & Promover uma visão
de mundo espiritual e normativa & Sermões, textos sagrados,
catequeses \\
\end{longtable}

\begin{quote}
(\ldots) a argumentação diz respeito a como melhor selecionar e
organizar argumentos de diferentes naturezas para alcançar objetivos
como demonstrar, persuadir e convencer.
\end{quote}

O caráter dialógico da argumentação, segundo Castro e outros (2019,
p.~14), depende do discurso em que essa argumentação deve se encaixar.
Tudo depende do contexto, portanto.

\begin{quote}
(\ldots) a argumentação se relaciona a públicos diversos (a quem se
destina), a objetos claros (o que está em questão) e a circunstâncias
específicas (em que momento e em que espaço se dá e de que modo se
realiza).
\end{quote}

É justamente o contexto que define o discurso do senso comum, que

\begin{itemize}
\item
  usa linguagem informal, coloquial e empírica;
\item
  é baseado na experiência cotidiana e em tradições;
\item
  tem pouca ou nenhuma verificação sistemática.
\end{itemize}

O objetivo do senso comum, portanto, é explicar e, às vezes, regular o
cotidiano com base em crenças compartilhadas.

Castro e outros (2019, p.~14), parafraseando Antonio Suárez Abreu,
lembram que

\begin{quote}
(\ldots) o senso comum é oriundo de variados discursos que formam o que
se chama de ``opinião pública''.
\end{quote}

\begin{quote}
Ela seria constituída (\ldots) por diversos discursos articulados que
permeiam toda a sociedade, independentemente de classe social.
\end{quote}

São exemplos de senso comum ditados populares, conselhos, ``sabedorias''
cotidianas.

De acordo com Savioli e Fiorin (\emph{apud} Castro et al., 2019, p.14)

\begin{quote}
(\ldots) os argumentos de senso comum normalmente são preconceituosos,
pois não são baseados em fatos e comprovações, mas em afirmações
usualmente generalizantes.
\end{quote}

\subsection{Questão 2}\label{questuxe3o-2-2}

Explique como cada ditado popular abaixo reproduz o discurso do senso
comum.

\begin{enumerate}
\def\labelenumi{\arabic{enumi}.}
\item
  Aqui se faz, aqui se paga.
\item
  Deus ajuda a quem cedo madruga.
\item
  Cada macaco no seu galho.
\item
  Roupa suja se lava em casa.
\item
  Mais vale um pássaro na mão do que dois voando.
\end{enumerate}

\begin{enumerate}
\def\labelenumi{\arabic{enumi}.}
\tightlist
\item
  \textbf{Aqui se faz, aqui se paga.}
\end{enumerate}

➤ Este ditado pressupõe uma espécie de justiça imediata ou natural,
segundo a qual toda ação negativa será punida com uma consequência
proporcional.

➤ É uma forma de senso comum porque simplifica a noção de justiça,
ignorando as complexidades éticas, sociais e jurídicas envolvidas em
cada situação.

\begin{enumerate}
\def\labelenumi{\arabic{enumi}.}
\setcounter{enumi}{1}
\tightlist
\item
  \textbf{Deus ajuda a quem cedo madruga.}
\end{enumerate}

➤ Atribui o sucesso pessoal ao esforço individual (acordar cedo), além
de invocar uma recompensa divina.

➤ Reproduz o senso comum ao naturalizar a ideia de meritocracia, sem
considerar fatores estruturais (como classe social, acesso à educação,
saúde etc.).

\begin{enumerate}
\def\labelenumi{\arabic{enumi}.}
\setcounter{enumi}{2}
\tightlist
\item
  \textbf{Cada macaco no seu galho.}
\end{enumerate}

➤ Defende que cada pessoa deve permanecer no seu lugar ou função,
evitando ``se meter'' em outras áreas.

➤ Sustenta um discurso de conservação da ordem social, reforçando
hierarquias e delimitando papéis de forma rígida e acrítica.

\begin{enumerate}
\def\labelenumi{\arabic{enumi}.}
\setcounter{enumi}{3}
\tightlist
\item
  \textbf{Roupa suja se lava em casa.}
\end{enumerate}

➤ Defende que conflitos familiares ou problemas íntimos devem ser
resolvidos em privado, sem exposição pública.

➤ Embora possa proteger a privacidade, favorece o silenciamento de
abusos e injustiças, sendo um exemplo clássico de senso comum que
reprime o debate público sobre questões sensíveis.

\begin{enumerate}
\def\labelenumi{\arabic{enumi}.}
\setcounter{enumi}{4}
\tightlist
\item
  \textbf{Mais vale um pássaro na mão do que dois voando.}
\end{enumerate}

➤ Sugere que é melhor manter algo garantido do que correr riscos em
busca de algo maior.

➤ Expressa um valor conservador, que inibe a ousadia e a inovação, e
parte de uma lógica pragmática e simplificadora, típica do senso comum.

O senso comum baseia-se na experiência cotidiana, transmitida
culturalmente, muitas vezes sem verificação crítica.

Castro e outros (2019, p.~15) afirmam que

\begin{quote}
(\ldots) o senso comum consiste no conhecimento vulgar, nas opiniões
diversas.
\end{quote}

\subsection{Questão 3}\label{questuxe3o-3-2}

Analise com atenção a imagem abaixo, extraída de um contexto cotidiano.
Identifique no texto da imagem as marcas do discurso do senso comum,
respondendo às perguntas abaixo:

\begin{itemize}
\item
  O enunciado generaliza alguma ideia?
\item
  O enunciado se baseia em evidências ou em crenças e emoções?
\item
  O enunciado reforça algum valor ou norma social amplamente aceita?
\end{itemize}

\textbf{1. O enunciado generaliza alguma ideia?}

Sim. A frase generaliza a figura materna, atribuindo a todas as mães um
papel idealizado como origem da vida e fonte inesgotável de amor.
Desconsidera, portanto, a diversidade de experiências familiares e
afetivas, incluindo mães ausentes, negligentes ou relações familiares
conflituosas.

\textbf{2. O enunciado se baseia em evidências ou em crenças e emoções?}

Baseia-se essencialmente em crenças e emoções. Não há comprovação
empírica ou referência a dados que sustentem a ideia de que o amor de
mãe seja eterno ou universal. A força da frase reside em seu apelo
afetivo, tocando valores subjetivos e culturais enraizados.

\textbf{3. O enunciado reforça algum valor ou norma social amplamente
aceita?}

Sim. A frase reforça a valorização idealizada da maternidade, um pilar
do senso comum em muitas culturas. Sustenta a noção de que mães devem
ser incondicionalmente amorosas, cuidadoras e abnegadas, consolidando
uma norma social tradicional de gênero.

O \textbf{senso crítico} é o oposto do senso comum, já que envolve
análise, comparação, verificação de evidências e compromisso com a
racionalidade e o método.

O ponto de vista científico se constrói em torno do senso crítico, do
pensamento científico.

\begin{center}\rule{0.5\linewidth}{0.5pt}\end{center}

\subsection{Questão 4}\label{questuxe3o-4-2}

Vamos estimular nossas habilidades cognitivas?

Pensar cientificamente e desenvolver o senso crítico não depende apenas
de acesso a informações, mas sobretudo da capacidade de organizar o
pensamento, identificar padrões, levantar hipóteses, interpretar dados e
tomar decisões fundamentadas.

Essas habilidades --- que envolvem atenção, memória, análise, síntese e
inferência --- são essenciais tanto para resolver problemas do cotidiano
quanto para compreender e produzir argumentos bem estruturados.

Para exercitá-las, proponho um desafio linguístico que exigirá de você
concentração, raciocínio lógico e uma boa dose de curiosidade. Em trios,
resolvam a questão abaixo:

Nesta árvore, os nomes dos personagens híbridos são formados a partir de
um padrão geral, com exceção de apenas dois. \textbf{Se} os nomes de
todos os Pokémon fossem formados seguindo o padrão geral, qual
\textbf{não} poderia ser um nome possível para o último Pokémon da
árvore (o mais de baixo)?

\begin{enumerate}
\def\labelenumi{\alph{enumi})}
\tightlist
\item
  krabgon
\item
  elekgon
\item
  spinpy
\item
  spinky
\item
  spinorb
\end{enumerate}

Resposta: (d)

Podemos notar que os nomes dos híbridos são formados pela combinação dos
nomes dos dois Pokémon que os originaram. Por exemplo, na primeira linha
temos:

\begin{verbatim}
•   shugon: formado por shu- de shuppet e -gon de bagon.
•   spinpip: formado por spin- de spinarak e -pip de hoppip.
•   kraborb: formado por krab- de krabby e -orb de voltorb.
•   elekpy: formado por elek- de elekid e -py de phanpy.
\end{verbatim}

A princípio, podemos pensar que o nome é sempre formado pela primeira
parte de um Pokémon com a última parte de outro. No entanto, isso não se
aplica a kraborb (a última parte de voltorb deveria ser -torb, e não
-orb). Portanto, a regra é mais simples: combinar o início de um nome
com o final de outro.

Na segunda linha, temos:

\begin{verbatim}
•   spingon: formado por spin- de spinpip e -gon de shugon.
•   kreky: formado por kr- de kraborb e -ky de elekpy.
\end{verbatim}

No caso de spingon, as partes dos nomes são as mesmas usadas
anteriormente. Como spinpip é a junção de spin- com -pip, spingon usa
spin- de spinarak, mantendo o padrão. Porém, kreky quebra o padrão, pois
não usou o final completo de elekpy (faltou o `p') e nem a parte inicial
correta de kraborb (deveria ser krab-). Assim, kreky é o primeiro nome
fora do padrão.

Depois temos:

\begin{verbatim}
•   krinky: formado por kr- de kreky e -gon de spingon.
\end{verbatim}

Esse nome também foge ao padrão, pois mistura as partes do meio (-in-)
de um nome com o início e o fim de outro. Portanto, krinky é o segundo
Pokémon fora do padrão.

Agora que identificamos os dois nomes fora do padrão, podemos descobrir
como seriam se seguissem a regra. A fusão de kraborb e elekpy deveria
usar a parte inicial de um com a parte final do outro, respeitando as
partes originais. Então, podemos ter:

\begin{verbatim}
•   kraborb + elekpy = krabpy ou
•   kraborb + elekpy = elekorb.
\end{verbatim}

Por fim, o nome do último Pokémon será a fusão de spingon com krabpy ou
elekorb, utilizando o mesmo princípio de combinar partes iniciais e
finais. As possíveis combinações são:

\begin{verbatim}
•   spingon + krabpy = spinpy ou krabgon;
•   spingon + elekorb = spinorb ou elekgon.
\end{verbatim}

Esses quatro nomes aparecem nas alternativas. O único que não seria
possível é spinky, pois ele depende de kreky, que já está fora do
padrão.

Um ponto interessante é que, ao seguir esse padrão de combinações, as
informações sobre os Pokémon originais podem se perder. Por exemplo, o
nome krabgon poderia parecer uma fusão de krabby e bagon, mas na verdade
é o resultado de várias fusões anteriores.

Por fim, o nome do problema, Poképu, é a junção de Pokémon e Khipu.

\begin{center}\rule{0.5\linewidth}{0.5pt}\end{center}

Você está presente?

\chapter{Discurso reportado. Gerenciamento de referências
bibliográficas}\label{discurso-reportado.-gerenciamento-de-referuxeancias-bibliogruxe1ficas}

Roteiro de aula elaborado no RStudio com o auxílio da inteligência
artificial ChatGPT, revisado e avaliado pelo autor antes de sua
publicação.

\section{Objetivos de aprendizagem}\label{objetivos-de-aprendizagem-3}

Ao final deste encontro e com base na leitura indicada, espera-se que
você seja capaz de:

\begin{itemize}
\item
  Identificar o que é discurso reportado e sua função na escrita
  acadêmica.
\item
  Diferenciar discurso direto e discurso indireto.
\item
  Instalar e configurar o Zotero com o conector do navegador.
\item
  Criar e organizar coleções de referências.
\item
  Adicionar itens à biblioteca a partir de sites e bases de dados.
\item
  Inserir e formatar citações e referências em Google Docs com o Zotero.
\item
  Utilizar o Zotero Sync para manter seus dados salvos na nuvem e
  acessíveis em diferentes dispositivos.
\end{itemize}

\begin{center}\rule{0.5\linewidth}{0.5pt}\end{center}

\section{Introdução}\label{introduuxe7uxe3o-2}

A escrita acadêmica exige clareza, coerência e, sobretudo, rigor na
apresentação das fontes de informação. As \textbf{normas de citação e
referência} --- como as da ABNT, APA ou Vancouver --- cumprem um papel
fundamental nesse processo: elas garantem a \emph{padronização}, a
\emph{transparência} e a \emph{credibilidade} do trabalho científico.

Um texto que segue essas normas permite ao leitor localizar com precisão
as obras consultadas, verificar os dados utilizados e reconhecer os
autores envolvidos na construção do conhecimento. A
\textbf{normalização}, portanto, não é apenas uma exigência formal, mas
um componente essencial da ética e da qualidade na produção acadêmica.

Parte fundamental dessa normalização diz respeito ao uso do
\textbf{discurso reportado}, ou seja, à forma como o \emph{autor do
texto acadêmico incorpora, comenta ou mobiliza a voz de outros autores
em sua escrita}. Essa prática envolve tanto o \textbf{discurso direto},
quando se reproduz fielmente as palavras de outrem, quanto o
\textbf{discurso indireto}, quando se parafraseia com as próprias
palavras.

Além de demonstrar domínio do conteúdo, o uso adequado do discurso
reportado insere o texto no campo científico, marcando sua filiação a
determinadas ideias, debates e tradições intelectuais. Isso permite ao
leitor reconhecer de onde vêm as informações, avaliar os fundamentos de
cada argumento e distinguir claramente a contribuição do autor.

Para apoiar esse processo, o \textbf{Zotero} é uma ferramenta gratuita e
de código aberto que funciona como um gerenciador de referências
bibliográficas. Ele permite salvar, organizar, citar e compartilhar
fontes de pesquisa de forma prática e automatizada, além de oferecer
sincronização em nuvem. Com ele, é possível aplicar estilos de citação
exigidos pelas instituições (como ABNT ou APA), o que facilita a
conformidade com as normas e contribui para a qualidade formal da
escrita acadêmica.

\subsection{Funções do discurso reportado na escrita
acadêmica}\label{funuxe7uxf5es-do-discurso-reportado-na-escrita-acaduxeamica}

\begin{verbatim}
•   Fundamentar teorias, conceitos e argumentos;
•   Dar credibilidade ao texto com base em autores reconhecidos;
•   Estabelecer relações dialógicas com outras pesquisas;
•   Evitar plágio, por meio da atribuição correta das ideias;
•   Aprofundar a análise crítica e o posicionamento do autor do texto.
\end{verbatim}

\subsection{Tipos de citações na escrita
acadêmica}\label{tipos-de-citauxe7uxf5es-na-escrita-acaduxeamica}

\begin{itemize}
\item
  Citação direta curta (até 3 linhas)

  • Deve ser integrada ao parágrafo, entre aspas duplas. • A fonte da
  citação (autor, ano e página) aparece entre parênteses. • Não deve
  conter itálico ou destaque gráfico.
\end{itemize}

Exemplo:

\begin{quote}
Segundo Amossy (2008, p.~9), ``todo discurso é atravessado por vozes e
saberes alheios''.
\end{quote}

\begin{quote}
Sabe-se que ``todo discurso é atravessado por vozes e saberes alheios''
(Amossy, 2008, p.~9).
\end{quote}

\begin{itemize}
\item
  Citação direta longa (mais de 3 linhas)

  • Deve ser destacada em parágrafo próprio, com: • Recuo de 4 cm da
  margem esquerda; • Fonte tamanho menor que a do corpo do texto; • Sem
  aspas; • Espaçamento simples entre as linhas; • Referência (autor,
  ano, página) após o ponto final da citação.
\end{itemize}

Exemplo:

\begin{quote}
O ethos discursivo, portanto, não é um traço individual do orador, mas
uma construção textual resultante da imagem que ele deseja projetar e da
imagem que o público está predisposto a aceitar (Amossy, 2008, p.~95).
\end{quote}

\begin{itemize}
\item
  Citação indireta

  • A ideia do autor é parafraseada, com menção à fonte sem aspas. • A
  indicação de página não é obrigatória, mas pode ser incluída.
\end{itemize}

Exemplo:

Bakhtin (2003) afirma que todo enunciado depende de outro anterior e que
não existe ponto de partida absoluto.

Citação de citação (apud) • Usada apenas quando a obra original não foi
consultada. • A citação deve mencionar os dois autores: o da ideia e o
da fonte consultada. • A expressão apud (em itálico) é obrigatória, com
os demais dados da fonte consultada.

Exemplo:

Para Vygotsky, o desenvolvimento do pensamento está ligado à linguagem
(apud OLIVEIRA, 2002, p.~37).

Atenção: o uso de apud deve ser excepcional. Sempre que possível,
consulte a fonte original.

\section{Aprendizagem prática}\label{aprendizagem-pruxe1tica-4}

\begin{enumerate}
\def\labelenumi{\arabic{enumi}.}
\item
  Crie uma coleção no Zotero com o nome: Ensino de Física.
\item
  Adicione cinco obras (artigos e/ou livros publicados entre 2024 e
  2025) à coleção, extraídas de uma busca no Google Acadêmico com os
  termos exatos ``clube de ciências'' e ``ensino de física''.
\item
  Crie um documento no Google Docs com duas seções (ambas formatadas
  como Título 1):

  • Introdução

  • Referências Bibliográficas
\item
  Insira citações das cinco obras na seção Introdução, utilizando o
  Zotero.
\item
  Gere automaticamente a lista de referências na seção Referências
  Bibliográficas, também com o Zotero.
\item
  Copie o conteúdo completo do documento e cole no campo abaixo
  (atividade individual).
\end{enumerate}

Preencha os campos abaixo para enviar o resultado da atividade
individual sobre o Zotero.

\chapter{Discurso reportado na escrita
acadêmica}\label{discurso-reportado-na-escrita-acaduxeamica}

Roteiro de aula elaborado no RStudio com o auxílio da inteligência
artificial ChatGPT, revisado e avaliado pelo professor antes de sua
publicação.

\section{Objetivos de aprendizagem}\label{objetivos-de-aprendizagem-4}

Ao final desta aula, espera-se que você seja capaz de:

\begin{itemize}
\tightlist
\item
  Reconhecer a função do discurso reportado na construção do
  conhecimento acadêmico;
\item
  Diferenciar discurso direto do discurso indireto;
\item
  Aplicar corretamente as normas da NBR 10520/2023 para citações diretas
  curtas, citações longas e de citações de citação;
\item
  Compreender o papel dos verbos \emph{dicendi} na introdução do
  discurso reportado.
\end{itemize}

Leitura indicada: Discurso reportado, capítulo do livro Escrever na
universidade 2 - texto e discurso, de Francisco Eduardo Vieira e Carlos
Alberto Faraco.

\href{https://drive.google.com/file/d/17QvXPqtVvgArZpXv0XJhTnwTayxVkxFc/view?usp=sharing}{Acesso
à leitura indicada}

\section{Introdução}\label{introduuxe7uxe3o-3}

A escrita acadêmica exige \emph{clareza}, \emph{coerência} e, sobretudo,
\emph{rigor} na apresentação das fontes de informação. As normas de
citação e referência --- como as da ABNT\footnote{A NBR 10520 trata de
  citações em documentos; a NBR 6023 trata de referências.} --- garantem
a padronização, a transparência e a credibilidade do trabalho
científico.

\subsection{Glossário}\label{glossuxe1rio}

\begin{quote}
Referência ``(\ldots) conjunto padronizado de elementos descritivos,
retirados de um documento, que permite sua identificação individual.''
(Associação Brasileira de Normas Técnicas, 2020)
\end{quote}

\begin{quote}
Citação ``(\ldots) menção de uma informação extraída de outra fonte''
(Associação Brasileira de Normas Técnicas, 2023)
\end{quote}

\textbf{Introdução}

\emph{Citações}

(Araújo; Farias, 2025) (Czolpinski; Brito; Raupp, 2024) (Azevedo;
Machado, 2024) (Jaime; Leonel, 2024) (Silva; Rosa; França, 2025)

\emph{Referências bibliográficas}

ARAÚJO, Jonas Vieira de; FARIAS, Cleilton Sampaio de. Aprendizagem
Experiencial no Ensino de Física: Uma Revisão sobre sua Aplicação na
Educação Profissional e Tecnológica. Rebena - Revista Brasileira de
Ensino e Aprendizagem, {[}s. l.{]}, v. 11, p.~13--31, 2025. Disponível
em: https://rebena.emnuvens.com.br/revista/article/view/309. Acesso em:
8 maio 2025.

CZOLPINSKI, Andrey de Lima; BRITO, Rafael da Costa; RAUPP, Daniele
Trajano. Clube de Ciências e a promoção da educação científica por meio
da extensão universitária. Revista Signos, {[}s. l.{]}, v. 45, n.~2,
2024. Disponível em:
https://www.univates.br/revistas/index.php/signos/article/view/3995.
Acesso em: 8 maio 2025.

AZEVEDO, Murillo Pereira; MACHADO, Letícia Sophia Rocha. Competências
digitais docentes no ensino de física\,: uma revisão sistemática da
literatura. {[}s. l.{]}, 2024. Disponível em:
https://lume.ufrgs.br/handle/10183/282037. Acesso em: 8 maio 2025.

JAIME, Danay Manzo; LEONEL, André Ary. Uso de simulações: Um estudo
sobre potencialidades e desafios apresentados pelas pesquisas da área de
ensino de física. Revista Brasileira de Ensino de Física, {[}s. l.{]},
v. 46, p.~e20230309, 2024. Disponível em:
https://www.scielo.br/j/rbef/a/PvcqYmVLssjYpggDb4Jmz8N/. Acesso em: 8
maio 2025.

SILVA, Dayane Cândido da; ROSA, Maria Izabella da Silva; FRANÇA, Suzane
Bezerra de. Formação docente em ciências biológicas no âmbito do
programa residência pedagógica na perspectiva do multiletramento.
Revista Ciências \& Ideias ISSN: 2176-1477, {[}s. l.{]}, p.~e25162772,
2025. Disponível em:
https://revistascientificas.ifrj.edu.br/index.php/reci/article/view/2772.
Acesso em: 8 maio 2025.

\begin{quote}
Segundo a ABNT, as referências bibliográficas deve ser dispostas em
ordem alfabética. Esteja atento à configuração do Zotero ou ao estilo
selecionado!
\end{quote}

\begin{quote}
Considere que, no Brasil, tipicamente se adota o sistema autor-data,
conforme a NBR 10520/2023 recomenda.
\end{quote}

\section{Aprendizagem prática}\label{aprendizagem-pruxe1tica-5}

\begin{enumerate}
\def\labelenumi{\arabic{enumi}.}
\tightlist
\item
  Observe, no texto\footnote{Texto totalmente inventado para os fins
    desta aula.} abaixo, as marcas de discurso reportado. Responda: qual
  informação no texto é original, isto é, produzido pelo autor do texto?
\end{enumerate}

\textbf{Ao discutir os desafios da escrita acadêmica no ensino superior,
Araújo e Farias (2025) destacam que} o maior obstáculo enfrentado pelos
estudantes é a dificuldade de articular argumentos com base em fontes
confiáveis, o que compromete a credibilidade de seus textos. \textbf{De
modo semelhante, Czolpinski, Brito e Raupp (2024) observam que} muitos
alunos ainda recorrem ao senso comum ou a fontes não verificadas, em vez
de dialogar com autores reconhecidos da área. \textbf{É nesse sentido
que Azevedo e Machado (2024) defendem} a necessidade de formar leitores
críticos, capazes de interpretar e mobilizar o conhecimento científico
de maneira ética e responsável. \textbf{Conforme afirmam Jaime e Leonel
(2024, p.~12),} ``a competência argumentativa não se desenvolve apenas
pela leitura, mas pela prática constante de escrever em diálogo com
outras vozes''. \textbf{Além disso, Silva, Rosa e França (2025) chamam a
atenção para} o papel dos professores como mediadores desse processo,
propondo atividades que estimulem o uso consciente de citações e a
construção de um posicionamento autoral no texto acadêmico.

Nenhuma informação do parágrafo analisado é original. Todo texto se
constrói como discurso de outro, portanto um discurso apenas reportado.

Marcar adequadamente o \textbf{discurso reportado} é essencial para:

\begin{itemize}
\tightlist
\item
  Situar a pesquisa na \textbf{rede de conhecimento em constante
  atualização};
\item
  Dialogar com o que já foi produzido sobre o tema;
\item
  Demonstrar respeito ao leitor, informando claramente o que é seu e o
  que é do outro;
\item
  Evitar o plágio, que é um \textbf{crime acadêmico}.
\end{itemize}

\begin{center}\rule{0.5\linewidth}{0.5pt}\end{center}

\section{Discurso reportado}\label{discurso-reportado}

É a \textbf{inserção do que outras pessoas disseram ou escreveram} em
seu texto acadêmico, permitindo ao autor:

\begin{itemize}
\tightlist
\item
  \textbf{Reforçar argumentos}, citando especialistas;
\item
  \textbf{Manifestar concordância} com outras pesquisas;
\item
  \textbf{Criticar ou refutar} ideias de outros autores.
\end{itemize}

\subsection{Discurso reportado direto}\label{discurso-reportado-direto}

Transcrição literal das palavras do outro, com as seguintes
características:

\begin{itemize}
\tightlist
\item
  \textbf{Entre aspas duplas} para citações curtas (até 3 linhas);
\item
  \textbf{Em bloco destacado}, sem aspas, para citações longas (mais de
  3 linhas);
\item
  \textbf{Indicação da autoria, ano e página};
\item
  Uso de \textbf{{[}sic{]}} para marcar erros existentes no original.
\end{itemize}

Exemplos

\textbf{Discurso reportado por meio de uma citação curta:}

\begin{quote}
Segundo Souza (2021, p.~17), ``a linguagem é o principal recurso para a
construção da realidade social''.
\end{quote}

\begin{quote}
Ao discutir os desafios da escrita acadêmica no ensino superior, Araújo
e Farias (2025) destacam que o maior obstáculo enfrentado pelos
estudantes é a dificuldade de articular argumentos com base em fontes
confiáveis, o que compromete a credibilidade de seus textos. De modo
semelhante, Czolpinski, Brito e Raupp (2024) observam que muitos alunos
ainda recorrem ao senso comum ou a fontes não verificadas, em vez de
dialogar com autores reconhecidos da área. É nesse sentido que Azevedo e
Machado (2024) defendem a necessidade de formar leitores críticos,
capazes de interpretar e mobilizar o conhecimento científico de maneira
ética e responsável. \textbf{Conforme afirmam Jaime e Leonel (2024,
p.~12), ``a competência argumentativa não se desenvolve apenas pela
leitura, mas pela prática constante de escrever em diálogo com outras
vozes''.} Além disso, Silva, Rosa e França (2025) chamam a atenção para
o papel dos professores como mediadores desse processo, propondo
atividades que estimulem o uso consciente de citações e a construção de
um posicionamento autoral no texto acadêmico.
\end{quote}

\textbf{Discurso reportado por meio de uma citação longa:}

\begin{quote}
Ao tratar da natureza social da linguagem, Araújo e Farias (2025, p.~45)
afirmam que A linguagem é uma prática social que constrói significados
coletivamente compartilhados. Nenhum enunciado surge isolado, mas em
resposta a outros enunciados, compondo uma rede de sentidos em constante
transformação. Essa rede é dinâmica e se atualiza a cada novo uso da
linguagem, revelando que todo dizer é, ao mesmo tempo, retomada e
recriação do que já foi dito. Assim, compreender a linguagem como
prática social é reconhecer seu papel na construção da realidade e na
negociação de sentidos entre os sujeitos.
\end{quote}

\begin{center}\rule{0.5\linewidth}{0.5pt}\end{center}

\subsection{Discurso reportado
indireto}\label{discurso-reportado-indireto}

Relato do que o outro disse, com as palavras do autor, indicando a
autoria e, opcionalmente, a página.

\begin{itemize}
\tightlist
\item
  Frequentemente introduzido por \textbf{verbos \emph{dicendi}}, como:
\item
  dizer, afirmar, explicar, justificar, sustentar, comentar, entre
  outros.
\end{itemize}

Exemplo

\begin{quote}
Segundo Souza (2021), a linguagem desempenha papel central na construção
da realidade social.
\end{quote}

\begin{center}\rule{0.5\linewidth}{0.5pt}\end{center}

\section{\texorpdfstring{Citação de citação
(\emph{apud})}{Citação de citação (apud)}}\label{citauxe7uxe3o-de-citauxe7uxe3o-apud}

Uso de uma fonte a partir de outra, quando não se teve acesso ao
documento original.

Exemplo

\begin{quote}
Segundo Bakhtin (1988, \emph{apud} Vieira e Faraco, 2019, p.~45), todo
enunciado é sempre dialógico.
\end{quote}

\begin{center}\rule{0.5\linewidth}{0.5pt}\end{center}

\begin{itemize}
\tightlist
\item
  Sinalize sempre o que é seu e o que é do outro;
\item
  Siga as normas da \textbf{NBR 10520/2023} e da \textbf{NBR 6023/2020},
  disponíveis no SIGAA (Biblioteca \textgreater{} Documentos ABNT);
\item
  Utilize o discurso reportado de forma estratégica para construir uma
  argumentação sólida e ética.
\end{itemize}

\section{Aprendizagem prática}\label{aprendizagem-pruxe1tica-6}

\begin{enumerate}
\def\labelenumi{\arabic{enumi}.}
\setcounter{enumi}{1}
\tightlist
\item
  Identifique no texto abaixo, publicado no periódico RenBio (Revista de
  Ensino de Biologia), todas as ocorrências de discurso reportado.
  Separe-os por tipo de citação: citação direta, citação indireta ou
  citação da citação. Copie e cole no documento à parte. Siga o modelo:
\end{enumerate}

Modelo:

\textbf{Citação indireta} Página 15: Silva e Andrade (2024) afirmam que
desenvolver a autonomia intelectual dos estudantes requer práticas
pedagógicas que estimulem a pesquisa e o pensamento crítico em todos os
níveis de ensino.

\textbf{Citação direta} Página 12: Percebe-se que ``os estudantes têm a
oportunidade de investigar problemas reais de suas comunidades, o que
torna o aprendizado mais significativo e socialmente relevante''
(Pereira; Costa; Nunes, 2025).

\textbf{Citação da citação} Página 43: Segundo Freire (1996, \emph{apud}
Souza; Almeida, 2025, p.~613), ``ensinar não é transferir conhecimento,
mas criar as possibilidades para a sua própria produção ou a sua
construção''.

\href{https://renbio.org.br/index.php/sbenbio/article/view/1347/497}{Texto
disponível aqui}

\textbf{Citação indireta}

Página 608: Estabelece-se, com isso, uma corresponsabilidade entre as
instituições de ensino superior e ensino básico com a formação inicial
de professores (Capes, 2018).

Página 608: As horas são divididas em três módulos de seis meses com
carga horária de 138 horas cada módulo (Capes, 2023).

Página 608: Segundo Nascimento e Bezerra (2019), o ensino de ciências
visa dar sentido e significado aos conhecimentos científicos e
tecnológicos que fazem parte do cotidiano.

Páginas 608-609: Segundo esses autores, é através da curiosidade e da
inquietação que os estudantes-clubistas são motivados a investigar,
questionar e experimentar, o que leva a uma compreensão mais profunda
dos conceitos científicos.

Página 609: O Clube de Ciências, segundo Milanesi e colaboradores
(2019), é um espaço de encontros com trocas de vivências, experimentos e
interações dinâmicas distintas das observadas em aulas convencionais,
acarretando desenvolvimento do conhecimento científico e do pensamento
crítico por meio de abordagens investigativas.

Página 609: Conforme delineado por Ramalho et al.~(2011), os Clubes de
Ciências representamespaços educativos nos quais estudantes,
compartilhando interesses em ciência, reúnem-se forado horário escolar
convencional.

Página 609: De acordo com Pires et al.~(2007), os Clubes de Ciências não
são meros espaços para aquisição de conhecimentos científicos e
tecnológicos por parte dos alunos.

Página 609: No contexto da Base Nacional Comum Curricular (BNCC), os
ateliês são espaços de aprendizado prático e interdisciplinar que
atendem às competências e habilidades da área relacionadas à BNCC,
promovendo a exploração, observação e a experimentação, valorizandoa
aprendizagem por meio de práticas investigativas, que são ferramentas
importantes para oestudo das ciências naturais (Brasil, 2018).

Página 613: Além disso, a colaboração entre o professor, os estudantes e
os demais envolvidos pode trazer contribuições significativas para a
alfabetização científica dos jovens, fortalecendo aspectos como a
assimilação de conceitos, a elaboração de modelos, o desenvolvimento de
habilidades cognitivas e o raciocínio científico (Buch; Schroeder,
2013).

Página 617: Estimular a curiosidade, a investigação e o pensamento
crítico entre os estudantes é crucial para formar cidadãos mais
preparados para os desafios do mundo atual, pois, segundo Buch e
Schroeder (2013), a construção do conhecimento científico é um processo
contínuo queenvolve a participação ativa do indivíduo.

\textbf{Citação direta}

Página 608: Dias e colaboradores (2013, p.~160) afirmam que: ``o ensino
de ciências deve estimular nos estudantes a capacidade deobservação,
despertando a curiosidade, a inquietação, a busca por novas respostas''.

Página 609: Nesse contexto, Duarte e Duarte (2013, p.~37) declaram que:

O ensino de ciências naturais não pode se limitar à promoção de mudanças
conceituais ou ao aprendizado do conhecimento científico. É necessário
também buscar mudanças metodológicas e de atitude nos alunos, bem como
ressignificar o ensino para construir um processo de aprendizagem, no
qualprofessores e alunos possam interagir de forma crítica e reflexiva
ao ensinarem e aprenderem (Duarte; Duarte, 2013, p.~37)

\textbf{Citação da citação}

Não há ocorrências.

\begin{enumerate}
\def\labelenumi{\arabic{enumi}.}
\setcounter{enumi}{2}
\tightlist
\item
  Na aula sobre gerenciamento de referências bibliográficas, numa
  aprendizagem prática, você foi solicitado a selecionar cinco artigos
  e/ou livros no Google Acadêmico, usando as palavras-chave ``ensino de
  física'' e ``clube de ciências''.
\end{enumerate}

Agora, usando as referências selecionadas por você, identifique qual é o
nome do periódico (revista) em que os artigos foram publicados e qual é
o nome da editora em que os livros foram publicados.

Você deverá expor oralmente quais são os periódicos e/ou editoras.

\texttt{r\ emo::ji("fire")} Lembre-se: na tarefa, não seriam aceitas
dissertações de mestrado, teses de doutorado ou artigos publicados em
anais de eventos. \texttt{r\ emo::ji("fire")}

\chapter{Fontes confiáveis de
informação}\label{fontes-confiuxe1veis-de-informauxe7uxe3o}

Roteiro de aula elaborado no RStudio com o auxílio da inteligência
artificial ChatGPT, revisado e avaliado pelo professor antes de sua
publicação.

\section{Objetivos de aprendizagem}\label{objetivos-de-aprendizagem-5}

Ao final desta aula, espera-se que você seja capaz de:

\begin{itemize}
\item
  Identificar fontes de informação confiáveis no meio acadêmico e as
  características que as tornam confiáveis.
\item
  Reconhecer como os artigos científicos são distribuídos em bases
  acadêmicas e bases livres.
\item
  Aplicar estratégias de busca de artigos científicos em bases de dados.
\end{itemize}

\phantomsection\label{Leitura}
Leitura indicada:

\textbf{Fontes confiáveis de informação acadêmica}, capítulo do livro
\textbf{Leitura e escrita acadêmicas}, de Nádia Studzinski Estima Castro
e colaboradores.

{[}🔗 Acesso à leitura
indicada{]}(https://bookshelf.vitalsource.com/books/9788533500228/page/11)

\section{Introdução}\label{introduuxe7uxe3o-4}

Na sociedade contemporânea, somos constantemente expostos a um grande
volume de informações, que circulam em diferentes formatos, mídias e
plataformas. Essa realidade exige de todos nós, especialmente no
ambiente acadêmico, o desenvolvimento de uma \textbf{postura crítica}
diante das \textbf{fontes de informação} que acessamos, utilizamos e
compartilhamos.

Segundo Castro e outros (2019, p.~43)

\begin{quote}
(\ldots) fontes de informação são os meios dos quais as pessoas retiram
o conteúdo de que precisam para sanar uma necessidade informacional
(\ldots).
\end{quote}

Fonte, nesse contexto, refere-se ao suporte em que um conteúdo está
registrado, seja ele um \emph{artigo científico}, um \emph{livro}, uma
\emph{tese}, uma \emph{dissertação}, um \emph{squib}, um \emph{resumo
expandido} publicado em anais de eventos acadêmicos ou outro tipo de
publicação reconhecida pela comunidade científica.

No contexto da pesquisa científica, essa responsabilidade é ainda maior,
pois a qualidade da produção acadêmica depende diretamente da
\textbf{confiabilidade das fontes} utilizadas para embasar argumentos,
justificar escolhas metodológicas e dialogar com a comunidade
científica.

Entender \textbf{o que caracteriza uma fonte confiável} é, portanto, uma
habilidade essencial para quem se dedica à pesquisa.

Fontes confiáveis são aquelas que passam por \textbf{processos
reconhecidos de validação do conhecimento}. Para identificá-las, o
pesquisador precisa atentar para os seguintes critérios.

\textbf{Autoridade}

\begin{itemize}
\item
  A fonte é proveniente de origem conhecida, como uma universidade, um
  governo ou um órgão público, ou vem de um indivíduo desconhecido?
\item
  Se for de indivíduos, é possível identificar o que mais publicaram e
  em qual editora? Possuem publicações em periódicos de alta qualidade
  revisados por especialistas?
\end{itemize}

\textbf{Precisão}

\begin{itemize}
\tightlist
\item
  A fonte foi revisada por especialistas no tema antes de ser aceita
  para publicação? Se é uma fonte de acesso on-line, faz referência a
  fontes e bibliografias utilizadas? Estão claras as datas de
  atualização do conteúdo?
\end{itemize}

\textbf{Viés e objetividade}

\begin{itemize}
\tightlist
\item
  A linguagem usada é lógica? Ou se utiliza de sensacionalismos e
  agressividade? Apresenta pontos de vista opostos ou defende uma
  ideologia? Faz referências a outras fontes? Como lida com a questão
  ética e a validade dos documentos?
\end{itemize}

\textbf{Cobertura}

\begin{itemize}
\tightlist
\item
  A fonte cobre toda a área de pesquisa a que se propõe ou necessita de
  maior amplitude, ou seja, maior desenvolvimento de seus argumentos?
\end{itemize}

\textbf{Atualidade}

\begin{itemize}
\tightlist
\item
  O conteúdo é recente e adequado à área do conhecimento? Considerando a
  área de conhecimento, a atualidade da informação é adequada? (Apesar
  de não ser regra geral, é indicado o período máximo de 5 anos a partir
  da publicação para garantir a atualidade dos dados em algumas áreas de
  pesquisa.)
\end{itemize}

\subsection{Bases de dados}\label{bases-de-dados}

Além disso, é fundamental conhecer os \textbf{diferentes tipos de bases
de dados e ferramentas de busca científica}, que oferecem acesso
organizado e qualificado à produção acadêmica nacional e internacional.

\begin{itemize}
\tightlist
\item
  \textbf{Bases de dados de acesso aberto}

  \begin{itemize}
  \tightlist
  \item
    Scielo
  \item
    BVS
  \item
    Portal de Periódicos da Capes
  \item
    Google Acadêmico (com ressalvas)
  \end{itemize}
\item
  \textbf{Bases de dados de acesso restrito}

  \begin{itemize}
  \tightlist
  \item
    ProQuest
  \item
    Springer Link
  \item
    Scopus (indexação e referências)
  \end{itemize}
\item
  \textbf{Repositórios e bibliotecas digitais}

  \begin{itemize}
  \tightlist
  \item
    BDTD (Biblioteca Digital de Teses e Dissertações)
  \item
    Catálogo de Teses e Dissertações da Capes
  \item
    Repositórios institucionais
  \end{itemize}
\end{itemize}

\section{Aprendizagem prática}\label{aprendizagem-pruxe1tica-7}

Objetivos

\begin{itemize}
\tightlist
\item
  Escolher, em equipe de quatro integrantes, um projeto de pesquisa da
  Ufersa como tema de estudo para a disciplina.
\item
  Localizar artigos científicos confiáveis relacionados a esse projeto,
  utilizando o Portal de Periódicos da Capes.
\end{itemize}

⚠️\textbf{Importante}⚠️ O artigo escolhido servirá de base inicial para
o seu plano de trabalho individual, que será avaliado ao final da
disciplina.

\subsection{✅ Etapa 1 - Escolha do tema da
equipe}\label{etapa-1---escolha-do-tema-da-equipe}

\begin{enumerate}
\def\labelenumi{\arabic{enumi}.}
\item
  Acessem os projetos de pesquisa em andamento na
  \href{https://sigaa.ufersa.edu.br/sigaa/public/pesquisa/consulta_projetos.jsf}{Ufersa}
\item
  Explorem, preferencialmente, os projetos do Departamento de Ciências
  Naturais, Matemática e Estatística (ou outro departamento de interesse
  da equipe).
\item
  Escolham um projeto que servirá de referência temática para os planos
  individuais dos membros da equipe.
\item
  Anotem as seguintes informações:
\end{enumerate}

\textbf{Nome do projeto}

\textbf{Nome do coordenador}

\textbf{Objetivo da pesquisa}

\textbf{Métodos de pesquisa}

\textbf{Palavras-chave}

\subsection{✅ Etapa 2 - Busca por fontes
confiáveis}\label{etapa-2---busca-por-fontes-confiuxe1veis}

\begin{enumerate}
\def\labelenumi{\arabic{enumi}.}
\tightlist
\item
  Acessem o Portal de Periódicos da Capes e localizem quatro artigos
  científicos (um para cada integrante da equipe) publicados nos últimos
  cinco anos que atendam aos seguintes critérios:
\end{enumerate}

\begin{itemize}
\item
  Publicados em periódicos classificados como Qualis A1 ou A2 da Capes,
  conforme a área do conhecimento.
\item
  Publicados em periódicos que adotam sistema de avaliação cega
  (double-blind peer review) e dupla avaliação por pareceristas
  independentes.
\item
  Disponíveis em texto completo no Portal da Capes.
\end{itemize}

\textbf{🔎 Como consultar a classificação Qualis de um periódico?}

O \textbf{Qualis Periódicos} é o sistema utilizado pela \textbf{Capes}
para classificar os periódicos científicos segundo critérios de
qualidade estabelecidos pelas áreas de avaliação da pós-graduação
brasileira. Essa classificação varia por área e por estrato, sendo
\textbf{A1} o nível mais alto.

📝 Passo a passo

\begin{enumerate}
\def\labelenumi{\arabic{enumi}.}
\item
  \textbf{Acesse a Plataforma Sucupira da Capes}\\
  \url{https://sucupira.capes.gov.br}
\item
  \textbf{No menu principal, clique em}\\
  \texttt{Consultas\ \textgreater{}\ Coleta\ \textgreater{}\ Qualis\ \textgreater{}\ Consulta\ Geral\ de\ Periódicos}
\item
  \textbf{Na página que abrir:}

  \begin{itemize}
  \tightlist
  \item
    Informe o \textbf{nome do periódico} ou o \textbf{ISSN} (se souber).
  \item
    Selecione a \textbf{área de avaliação} (exemplo: Ensino, Física,
    Educação).
  \item
    Informe o \textbf{Ano da Avaliação} (utilize a última disponível,
    geralmente 2017-2020).
  \item
    Clique em \textbf{``Pesquisar''}.
  \end{itemize}
\item
  \textbf{Verifique os resultados:}

  \begin{itemize}
  \tightlist
  \item
    O sistema exibirá o periódico com a respectiva \textbf{classificação
    Qualis} para a área e o ano selecionados.
  \item
    ⚠️ Um mesmo periódico pode ter classificações diferentes em áreas
    distintas.
  \end{itemize}
\end{enumerate}

📌 Importante

\begin{itemize}
\tightlist
\item
  Verifique sempre a \textbf{área correta}, relacionada ao seu projeto
  ou tema de estudo.
\item
  Prefira periódicos classificados como \textbf{A1 ou A2}, que indicam
  alta qualidade segundo os critérios da Capes.
\end{itemize}

\begin{enumerate}
\def\labelenumi{\arabic{enumi}.}
\setcounter{enumi}{1}
\tightlist
\item
  Após a escolha dos artigos, levantem as seguintes informações sobre os
  periódicos em que esses artigos estão publicados:
\end{enumerate}

\textbf{Título do periódico}

\textbf{Classificação Qualis (A1 ou A2) e área do conhecimento}

\textbf{Instituição, sociedade científica ou editora responsável pelo
periódico}

\textbf{Periodicidade de publicação} (mensal, trimestral, semestral
etc.)

\textbf{Sistema de avaliação adotado} (simples cega, dupla cega, aberta,
etc.)

\textbf{Bases de indexação} (ex.: Scopus, Web of Science, etc.)

\textbf{Link oficial do periódico}

📌 Organização no Zotero

Após localizar e escolher o artigo, \textbf{cada integrante da equipe
deve adicionar o artigo selecionado ao seu Zotero}, criando uma
\textbf{coleção chamada ``Iniciação científica''}.

Isso garantirá que você organize suas fontes de maneira adequada para
usá-las nas próximas etapas da disciplina.

Se você ainda não instalou ou configurou o Zotero, consulte o tutorial
disponível no Bookdown da disciplina.

🔗 \hyperref[]{Acesse o tutorial de instalação e uso do Zotero}\\
\emph{(https://profmariomartins.github.io/anete\_suporte/zotero.html)}

\subsection{✅ Etapa 3 - Compartilhamento dos
achados}\label{etapa-3---compartilhamento-dos-achados}

Cada integrante da equipe deve postar o resultado da sua escolha no
Padlet da disciplina, utilizando o seguinte formato:

\textbf{Artigo escolhido}

\begin{itemize}
\tightlist
\item
  Título do artigo: {[}Título do artigo{]}
\item
  Link do artigo: {[}Link para o artigo completo{]}
\end{itemize}

\textbf{Informações sobre o periódico}

\begin{itemize}
\tightlist
\item
  Título do periódico: {[}Nome do periódico{]}
\item
  Classificação Qualis: {[}A1 ou A2{]} - Área: {[}Área do
  conhecimento{]}
\item
  Instituição responsável: {[}Universidade/Editora/Sociedade
  científica{]}
\item
  Periodicidade: {[}Mensal/Trimestral/Semestral{]}
\item
  Sistema de avaliação: {[}Simples cega/Dupla cega/Outra{]}
\item
  Bases de indexação: {[}Exemplo: Scopus, Web of Science, etc.{]}
\item
  Link oficial do periódico: {[}URL{]}
\end{itemize}

🔗 Acesse o Padlet da turma
\href{https://padlet.com/mariomartins/anete-t01-b3r3eohs4gd7np8m}{aqui}

Você deve fazer login no Padlet para que a postagem seja identificada.

\chapter{Revisão de conteúdos}\label{revisuxe3o-de-conteuxfados}

\part{Unidade II}

\chapter{A definir}\label{a-definir}

\part{Unidade III}

\chapter{Título da aula}\label{tuxedtulo-da-aula}

Roteiro de aula elaborado no RStudio com o auxílio da inteligência
artificial ChatGPT, revisado e avaliado pelo professor antes de sua
publicação.

\section{Objetivos de aprendizagem}\label{objetivos-de-aprendizagem-6}

\begin{itemize}
\tightlist
\item
  Objetivo 1
\item
  Objetivo 2
\item
  Objetivo 3
\end{itemize}

\phantomsection\label{Leitura}
Leitura indicada:

\textbf{Texto}, capítulo do livro \textbf{Capítulo}, de Autores.

{[}🔗 Acesso à leitura indicada{]}(https\ldots)

\section{Introdução}\label{introduuxe7uxe3o-5}

\section{Aprendizagem prática}\label{aprendizagem-pruxe1tica-8}

\subsection{Questão 1}\label{questuxe3o-1-4}

Resposta

\subsection{Questão 2}\label{questuxe3o-2-3}

Resposta

⚠️\textbf{Importante}⚠️

\textbf{Quadro de instruções}

\section{Encerramento}\label{encerramento}

\part{Tutoriais}

\chapter{Zotero: Gerenciador de referências
bibliográficas}\label{zotero-gerenciador-de-referuxeancias-bibliogruxe1ficas}

\chapter{Tutorial para iniciantes no
Zotero}\label{tutorial-para-iniciantes-no-zotero}

Este tutorial apresenta o passo a passo básico para começar a usar o
\textbf{Zotero}, um gerenciador de referências bibliográficas gratuito e
de código aberto.

\section{1. Baixando e instalando o
Zotero}\label{baixando-e-instalando-o-zotero}

\begin{itemize}
\item
  \textbf{Acesse o site oficial:}\\
  \url{https://www.zotero.org/download}
\item
  \textbf{Escolha seu sistema operacional:}

  \begin{itemize}
  \tightlist
  \item
    \textbf{MacOS:} Clique em \emph{Download for macOS}. Após o
    download, abra o instalador e arraste o Zotero para a pasta
    \textbf{Aplicativos}.
  \item
    \textbf{Windows:} Clique em \emph{Download for Windows}. Execute o
    instalador (.exe) e siga as instruções na tela.
  \item
    \textbf{Linux:} Clique em \emph{Download for Linux}. Baixe a versão
    adequada, extraia a pasta e execute o arquivo Zotero, ou siga as
    instruções específicas para sua distribuição (ex.: terminal no
    Ubuntu).
  \end{itemize}
\end{itemize}

\section{2. Instalando o Zotero Connector no
navegador}\label{instalando-o-zotero-connector-no-navegador}

\begin{itemize}
\tightlist
\item
  \textbf{No mesmo site}, clique em \emph{Install Zotero Connector}.
\item
  Você será direcionado para a loja de extensões do seu navegador
  (Chrome, Firefox ou Edge).
\item
  Clique em \textbf{Adicionar ao navegador} e confirme a instalação.
\item
  O ícone do Zotero (uma folha com a letra Z) aparecerá próximo à barra
  de endereços do navegador.
\end{itemize}

\section{3. Definindo o estilo de citação e
referência}\label{definindo-o-estilo-de-citauxe7uxe3o-e-referuxeancia}

\begin{itemize}
\tightlist
\item
  \textbf{Abra o Zotero}.
\item
  No menu superior, clique em:

  \begin{itemize}
  \tightlist
  \item
    \textbf{Zotero \textgreater{} Preferences (macOS)}
  \item
    \textbf{Edit \textgreater{} Preferences (Windows/Linux)}
  \end{itemize}
\item
  Selecione a aba \textbf{Cite}, depois a subaba \textbf{Styles}.
\item
  Escolha o estilo desejado (no Brasil, recomenda-se \textbf{ABNT}).
\item
  Se não estiver listado, clique em \textbf{Get additional styles},
  pesquise por ``ABNT'' e instale:

  \begin{itemize}
  \tightlist
  \item
    \emph{Universidade Federal do Rio Grande do Sul - ABNT (autoria
    completa) (Português - Brasil)}.
  \end{itemize}
\end{itemize}

\section{4. Criando coleções e adicionando
Itens}\label{criando-coleuxe7uxf5es-e-adicionando-itens}

\subsection{Crie uma coleção}\label{crie-uma-coleuxe7uxe3o}

\begin{itemize}
\tightlist
\item
  Clique com o botão direito em \textbf{Minha Biblioteca} e selecione
  \textbf{Nova coleção}.
\item
  Nomeie a coleção (ex.: \emph{Projeto Argumentação}) e pressione
  \textbf{Enter}.
\end{itemize}

\subsection{Adicione um item
manualmente}\label{adicione-um-item-manualmente}

\begin{itemize}
\tightlist
\item
  Com a coleção aberta, clique no botão \textbf{+ (Novo Item)} na barra
  superior.
\item
  Escolha o tipo de material (Artigo, Livro, Tese, etc.).
\item
  Preencha os campos na coluna da direita: título, autor(es), ano,
  editora, DOI, URL, etc.
\end{itemize}

\subsection{Adicione um item
automaticamente}\label{adicione-um-item-automaticamente}

\begin{itemize}
\tightlist
\item
  Acesse um artigo em sites como \textbf{Google Acadêmico} ou
  \textbf{SciELO}.
\item
  Clique no \textbf{ícone do Zotero Connector} no navegador.
\item
  Selecione a coleção de destino.
\item
  O item será salvo automaticamente com os metadados.
\end{itemize}

\section{5. Inserindo citações e referências em documentos no Google
Docs}\label{inserindo-citauxe7uxf5es-e-referuxeancias-em-documentos-no-google-docs}

\begin{quote}
💡 \textbf{Pré-requisito:} Mantenha o Zotero aberto durante o processo.
\end{quote}

\subsection{Insira citação}\label{insira-citauxe7uxe3o}

\begin{itemize}
\tightlist
\item
  Abra um documento no \textbf{Google Docs}.
\item
  Clique no menu \textbf{Zotero} que aparece na barra do Google Docs.
\item
  Selecione \textbf{Add/Edit Citation}.
\item
  Digite o título, autor ou ano, escolha o item e pressione
  \textbf{Enter}.
\end{itemize}

\subsection{Insira bibliografia}\label{insira-bibliografia}

\begin{itemize}
\tightlist
\item
  No final do documento, clique novamente em \textbf{Zotero}.
\item
  Selecione \textbf{Add/Edit Bibliography}.
\item
  A lista de referências será gerada automaticamente com base nas
  citações inseridas.
\end{itemize}

\section{6. Sincronizando com a nuvem do
Zotero}\label{sincronizando-com-a-nuvem-do-zotero}

\subsection{Crie uma conta Zotero}\label{crie-uma-conta-zotero}

\begin{itemize}
\tightlist
\item
  Acesse \url{https://www.zotero.org/user/register}.
\item
  Preencha os campos solicitados e confirme o cadastro pelo e-mail.
\end{itemize}

\subsection{Configure a
sincronização}\label{configure-a-sincronizauxe7uxe3o}

\begin{itemize}
\tightlist
\item
  Abra o Zotero e vá em \textbf{Zotero \textgreater{} Preferences
  \textgreater{} Sync}.
\item
  Insira seu \textbf{e-mail e senha} da conta Zotero.
\item
  Ative as opções \textbf{Sync automatically} e \textbf{Sync full-text
  content}.
\item
  Clique em \textbf{Set up syncing} e depois em \textbf{OK}.
\end{itemize}

\subsection{Confirme o backup online}\label{confirme-o-backup-online}

\begin{itemize}
\tightlist
\item
  Acesse \url{https://www.zotero.org} e faça login.
\item
  Clique em \textbf{Web Library} no menu superior.
\item
  Verifique se as coleções e os itens estão sincronizados corretamente.
\end{itemize}

\begin{center}\rule{0.5\linewidth}{0.5pt}\end{center}

\subsection{📌 Algumas
considerações\ldots{}}\label{algumas-considerauxe7uxf5es}

\begin{itemize}
\tightlist
\item
  Organize suas referências de forma prática.
\item
  Garanta a correta formatação de citações e referências conforme normas
  da ABNT.
\item
  Sincronize sua biblioteca entre diferentes dispositivos.
\item
  Para mais informações, acesse a
  \href{https://www.zotero.org/support/}{Documentação Oficial do
  Zotero}.
\end{itemize}

\chapter{Portal de periódicos da
Capes}\label{portal-de-periuxf3dicos-da-capes}

\chapter{Tutorial para iniciantes no Portal de Periódicos
Capes}\label{tutorial-para-iniciantes-no-portal-de-periuxf3dicos-capes}

Este tutorial irá guiá-lo pelos passos básicos para começar suas
pesquisas no Portal de Periódicos Capes, explorando as diferentes formas
de encontrar materiais para seu trabalho acadêmico.

\section{1. Acessando o Portal (tipos de
acesso)}\label{acessando-o-portal-tipos-de-acesso}

\begin{itemize}
\item
  \textbf{Como Acessar:}\\
  Abra o Google e digite \emph{``Periódicos Capes''}. O site correto é
  \url{https://www.periodicos.Capes.gov.br}.
\item
  \textbf{Tipos de Conteúdo/Acesso:}

  \begin{itemize}
  \tightlist
  \item
    \textbf{Conteúdo livre:} disponível para qualquer pessoa.
  \item
    \textbf{Conteúdo assinado:} disponível para os IPs identificados das
    instituições participantes. Caso você esteja acessando fora da rede
    da sua instituição, é necessário efetuar o login na Comunidade
    Acadêmica Federada (CAFe). A Ufersa é instituição parceira).

    \begin{itemize}
    \tightlist
    \item
      Para acessar o conteúdo assinado, clique em \textbf{``Acesso
      CAFE''}, selecione sua instituição e insira o login e a senha
      fornecidos pela universidade.
    \item
      No canto esquerdo, você verá se está conectado à sua instituição
      ou em acesso livre.
    \end{itemize}
  \end{itemize}
\end{itemize}

\section{2. ``Meu Espaço''}\label{meu-espauxe7o}

\begin{itemize}
\tightlist
\item
  Qualquer pessoa pode se cadastrar em \textbf{``Meu Espaço''}, mesmo
  sem vínculo institucional.
\item
  Isso permite salvar histórico de buscas e acessar funcionalidades
  adicionais.
\end{itemize}

\section{3. Explorando as opções de
busca}\label{explorando-as-opuxe7uxf5es-de-busca}

\begin{itemize}
\tightlist
\item
  Clique em \textbf{``Acervo''} no menu principal.
\item
  As opções principais são:

  \begin{enumerate}
  \def\labelenumi{\arabic{enumi}.}
  \tightlist
  \item
    \textbf{Buscar Assunto}
  \item
    \textbf{Listar Bases e Coleções}
  \item
    \textbf{Listar Livros}
  \item
    \textbf{Listar Periódicos}
  \end{enumerate}
\end{itemize}

\section{4. Busca simples por assunto}\label{busca-simples-por-assunto}

\begin{itemize}
\tightlist
\item
  Clique em \textbf{``Buscar Assunto''}.
\item
  Digite o tema (ex.: \emph{teletrabalho}).
\item
  Clique na lupa.
\item
  Nos resultados:

  \begin{itemize}
  \tightlist
  \item
    Cadeado verde = acesso aberto.
  \item
    Clique em \textbf{``acessar''} para ler ou baixar o material.
  \end{itemize}
\end{itemize}

\section{5. Busca avançada e
específica}\label{busca-avanuxe7ada-e-especuxedfica}

\begin{itemize}
\tightlist
\item
  Na tela de resultados, clique em \textbf{``Busca Avançada''}.
\item
  Configure os campos e use operadores booleanos:

  \begin{itemize}
  \tightlist
  \item
    \textbf{AND}: restringe (ex.: \emph{``teletrabalho AND servidores
    públicos''}).
  \item
    \textbf{OR}: amplia (ex.: \emph{``teletrabalho OR home office''}).
  \item
    \textbf{NOT}: exclui (ex.: \emph{``teletrabalho NOT servidores
    públicos''}).
  \item
    \textbf{Aspas (``\,``)}: busca exata (ex.: \emph{``servidores
    públicos''}).
  \end{itemize}
\item
  Aplique filtros na lateral esquerda:

  \begin{itemize}
  \tightlist
  \item
    Tipo de acesso (livre ou assinado)
  \item
    Tipo de recurso (artigo, livro, etc.)
  \item
    Data de publicação
  \item
    Área do conhecimento
  \item
    Idioma
  \end{itemize}
\end{itemize}

\section{6. Explorando bases, livros e
periódicos}\label{explorando-bases-livros-e-periuxf3dicos}

\begin{itemize}
\tightlist
\item
  \textbf{Listar Bases e Coleções:} Repositórios e bibliotecas digitais.
\item
  \textbf{Listar Livros:} Busque livros por título ou ISBN.
\item
  \textbf{Listar Periódicos:} Navegue por revistas científicas
  específicas.
\end{itemize}

\begin{center}\rule{0.5\linewidth}{0.5pt}\end{center}

\subsection{📌 Algumas
considerações\ldots{}}\label{algumas-considerauxe7uxf5es-1}

\begin{itemize}
\tightlist
\item
  O Portal está em constante atualização.
\item
  Combine buscas básicas, avançadas e filtros para melhores resultados.
\item
  Explore repositórios institucionais e revistas específicas além da
  busca por assunto.
\item
  Para dúvidas ou orientações, consulte a equipe da biblioteca da sua
  instituição.
\end{itemize}

\part{Modelos}

\chapter{Plano de trabalho}\label{plano-de-trabalho}

Modelo de plano de trabalho

\begin{tcolorbox}[enhanced jigsaw, opacityback=0, colback=white, breakable, opacitybacktitle=0.6, leftrule=.75mm, left=2mm, colframe=quarto-callout-tip-color-frame, colbacktitle=quarto-callout-tip-color!10!white, coltitle=black, arc=.35mm, toprule=.15mm, bottomtitle=1mm, toptitle=1mm, titlerule=0mm, title=\textcolor{quarto-callout-tip-color}{\faLightbulb}\hspace{0.5em}{Dica}, rightrule=.15mm, bottomrule=.15mm]

Este é um modelo de plano de trabalho para a iniciação científica --- um
tipo de proposta de pesquisa que se vincula a projetos já cadastrados
por professores da Ufersa.

Pelas regras institucionais, quem submete oficialmente o plano para
concorrer a bolsas de Iniciação Científica é o(a) professor(a)
coordenador(a) do projeto. Mas isso não significa que você não possa
propor uma pesquisa.

Se você tem uma ideia de pesquisa e gostaria de ser bolsista de IC, você
pode se antecipar e apresentar sua proposta ao professor responsável por
um projeto da sua área.

Essa atitude mostra iniciativa, organização e compromisso com a ciência
--- qualidades que contam muito em qualquer etapa da vida acadêmica.

🔎 Para descobrir quais projetos de pesquisa estão em execução na
Ufersa, clique
\href{https://sigaa.ufersa.edu.br/sigaa/public/pesquisa/consulta_projetos.jsf}{aqui}.

\end{tcolorbox}

Título

Título conciso e informativo da proposta de trabalho do estudante.

Introdução e justificativa

Apresente o contexto da proposta e sua relevância. Indique como ela se
vincula ao projeto de pesquisa ao qual se integra. Justifique a
importância do estudo com base em evidências (dados, autores,
relatórios).

Objetivos

Objetivo geral

Descrever o objetivo principal da proposta.

Objetivos específicos

\begin{itemize}
\tightlist
\item
  Objetivo 1
\item
  Objetivo 2
\item
  Objetivo 3
\end{itemize}

Metodologia

Descreva os procedimentos que serão adotados ao longo do plano de
trabalho. Inclua informações sobre instrumentos, etapas, técnicas de
análise e, quando pertinente, sobre aprovação ética (CEP/CEUA).

Habilidades a serem desenvolvidas

Liste as competências que você pretende desenvolver:

\begin{itemize}
\tightlist
\item
  Habilidade 1 (ex: análise de dados)
\item
  Habilidade 2 (ex: escrita acadêmica)
\item
  Habilidade 3 (ex: uso de ferramentas digitais)
\end{itemize}

Cronograma de atividades

\begin{longtable}[]{@{}
  >{\raggedright\arraybackslash}p{(\linewidth - 14\tabcolsep) * \real{0.4776}}
  >{\centering\arraybackslash}p{(\linewidth - 14\tabcolsep) * \real{0.0746}}
  >{\centering\arraybackslash}p{(\linewidth - 14\tabcolsep) * \real{0.0746}}
  >{\centering\arraybackslash}p{(\linewidth - 14\tabcolsep) * \real{0.0746}}
  >{\centering\arraybackslash}p{(\linewidth - 14\tabcolsep) * \real{0.0746}}
  >{\centering\arraybackslash}p{(\linewidth - 14\tabcolsep) * \real{0.0746}}
  >{\centering\arraybackslash}p{(\linewidth - 14\tabcolsep) * \real{0.0746}}
  >{\centering\arraybackslash}p{(\linewidth - 14\tabcolsep) * \real{0.0746}}@{}}
\toprule\noalign{}
\begin{minipage}[b]{\linewidth}\raggedright
Atividades
\end{minipage} & \begin{minipage}[b]{\linewidth}\centering
Set
\end{minipage} & \begin{minipage}[b]{\linewidth}\centering
Out
\end{minipage} & \begin{minipage}[b]{\linewidth}\centering
Nov
\end{minipage} & \begin{minipage}[b]{\linewidth}\centering
Dez
\end{minipage} & \begin{minipage}[b]{\linewidth}\centering
Jan
\end{minipage} & \begin{minipage}[b]{\linewidth}\centering
Fev
\end{minipage} & \begin{minipage}[b]{\linewidth}\centering
Mar
\end{minipage} \\
\midrule\noalign{}
\endhead
\bottomrule\noalign{}
\endlastfoot
Leitura e revisão de literatura & X & X & & & & & \\
Coleta de dados & & X & X & & & & \\
Tratamento dos dados & & & X & X & & & \\
Análise dos dados & & & & X & X & & \\
Elaboração de relatório parcial & & & & & X & X & \\
Escrita do relatório final & & & & & & X & X \\
Apresentação dos resultados & & & & & & & X \\
\end{longtable}

Referências

SOBRENOME, Nome. \emph{Título da obra}. Local: Editora, ano.

SOBRENOME, Nome. ``Título do artigo''. \emph{Nome do periódico}, v. X,
n.~Y, p.~Z--Z, ano.




\end{document}
